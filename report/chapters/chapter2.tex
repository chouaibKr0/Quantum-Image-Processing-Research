% chapters/chapter2.tex
% Chapter 2: Quantum Clustering - Main file that includes algorithm subsections

\chapter{Quantum Clustering: A New Frontier in Image Segmentation}

\section{Introduction: Bridging Quantum Mechanics and Computer Vision}

The intersection of quantum computing and computer vision represents an emerging field with significant potential for advancing image segmentation capabilities~\cite{wang2022review,cai2018survey,aburaed2017advances}. This chapter explores how quantum mechanical principles can be leveraged to develop more efficient clustering algorithms for image analysis~\cite{ruan2021qip,chakraborty2022challenges}.

Quantum computing fundamentally differs from classical computing in its ability to exploit quantum mechanical phenomena—superposition, entanglement, and interference—to process information~\cite{sood2024quantum}. These properties enable quantum algorithms to explore solution spaces in ways that are impossible for classical computers~\cite{li2020quantum,ramezani2020mlquantum}.

\begin{keyconceptbox}[Quantum Computing for Clustering]
The key insight driving quantum clustering research is that many clustering problems can be reformulated as~\cite{aimeur2007clustering,xu2015clustering}:
\begin{itemize}
    \item Distance estimation problems (quantum advantage via amplitude encoding)
    \item Eigenvalue problems (quantum advantage via quantum phase estimation)
    \item Optimization problems (quantum advantage via QAOA or quantum annealing)
\end{itemize}
\end{keyconceptbox}

\subsection{The NISQ Era and Its Implications}

Current quantum computers operate in the Noisy Intermediate-Scale Quantum (NISQ) era, characterized by:

\begin{itemize}
    \item \textbf{Limited qubit counts:} Typically 50-1000 qubits
    \item \textbf{High error rates:} Gate errors of $10^{-3}$ to $10^{-2}$
    \item \textbf{Limited coherence times:} Microseconds to milliseconds
    \item \textbf{Restricted connectivity:} Not all qubits can interact directly
\end{itemize}

\begin{warningbox}
These limitations mean that theoretically optimal quantum algorithms may not be practical on current hardware. Much research focuses on variational and hybrid classical-quantum approaches that are more noise-tolerant.
\end{warningbox}

\subsection{From Classical to Quantum: Key Transformations}

Applying quantum computing to image segmentation requires several key transformations:

\begin{enumerate}
    \item \textbf{Data encoding:} Converting classical image data into quantum states
    \item \textbf{Algorithm design:} Developing quantum circuits that perform clustering operations
    \item \textbf{Measurement and interpretation:} Extracting classical cluster assignments from quantum measurements
\end{enumerate}

\section{Approaches to Quantum Image Representation and Processing}

A fundamental challenge in quantum image processing is encoding classical image data into quantum states~\cite{yan2014qip}. Several representations have been proposed, each with distinct advantages for different applications~\cite{cai2018survey,wang2022review}.

\subsection{Flexible Representation of Quantum Images (FRQI)}

FRQI encodes a $2^n \times 2^n$ grayscale image using $2n + 1$ qubits:

\begin{definitionbox}[FRQI Representation]
\begin{equation}
    \ket{\text{FRQI}} = \frac{1}{2^n} \sum_{i=0}^{2^{2n}-1} \left(\cos\theta_i\ket{0} + \sin\theta_i\ket{1}\right) \otimes \ket{i}
\end{equation}
where $\theta_i \in [0, \pi/2]$ encodes the grayscale value of pixel $i$, and $\ket{i}$ encodes the pixel position.
\end{definitionbox}

\textbf{Advantages:}
\begin{itemize}
    \item Compact representation: $2n + 1$ qubits for $2^{2n}$ pixels
    \item Natural encoding for grayscale images
\end{itemize}

\textbf{Limitations:}
\begin{itemize}
    \item Complex state preparation requiring $O(2^{2n})$ gates
    \item Limited grayscale precision (single qubit for intensity)
\end{itemize}

\subsection{Novel Enhanced Quantum Representation (NEQR)}

NEQR improves upon FRQI by using a basis encoding for pixel intensities:

\begin{definitionbox}[NEQR Representation]
\begin{equation}
    \ket{\text{NEQR}} = \frac{1}{2^n} \sum_{y=0}^{2^n-1} \sum_{x=0}^{2^n-1} \ket{f(y,x)}\ket{yx}
\end{equation}
where $\ket{f(y,x)} = \ket{c_0^{yx}c_1^{yx}\ldots c_{q-1}^{yx}}$ is the $q$-bit binary representation of the grayscale value at position $(y,x)$.
\end{definitionbox}

\textbf{Advantages:}
\begin{itemize}
    \item Higher precision: $q$ bits for intensity values (e.g., $q=8$ for 256 levels)
    \item Simpler image operations through bit-level manipulations
    \item More efficient for certain quantum image processing operations
\end{itemize}

\subsection{Amplitude Encoding for Feature Vectors}

For clustering applications, amplitude encoding is particularly important as it allows efficient representation of high-dimensional feature vectors:

\begin{definitionbox}[Amplitude Encoding]
A normalized vector $\mathbf{x} = (x_0, x_1, \ldots, x_{N-1})^T$ with $\|\mathbf{x}\| = 1$ can be encoded as:
\begin{equation}
    \ket{\mathbf{x}} = \sum_{i=0}^{N-1} x_i \ket{i}
\end{equation}
requiring only $\lceil\log_2 N\rceil$ qubits to represent $N$ amplitudes.
\end{definitionbox}

\begin{tipbox}
Amplitude encoding enables exponential compression: a feature vector with $N = 2^n$ components requires only $n$ qubits. This is crucial for encoding high-dimensional image features efficiently.
\end{tipbox}

\subsection{Quantum Distance Estimation}

A key operation for clustering is computing distances between data points. Quantum computers can estimate distances using the swap test:

\begin{equation}
    |\braket{\mathbf{x}|\mathbf{y}}|^2 = 1 - \frac{d(\mathbf{x}, \mathbf{y})^2}{2}
\end{equation}

where $d(\mathbf{x}, \mathbf{y})$ is the Euclidean distance between normalized vectors.

The swap test circuit measures the overlap between two quantum states with $O(1)$ quantum operations (after state preparation), compared to $O(d)$ classical operations for $d$-dimensional vectors.

\section{A Comprehensive Review of Quantum Segmentation Algorithms}

This section presents a detailed analysis of quantum clustering algorithms that have been proposed for image segmentation. Rather than an exhaustive catalog, we focus on representative methods that illustrate the diverse strategies for leveraging quantum computation in clustering tasks.

\subsection{Organization and Scope}

The algorithms reviewed here are organized according to their underlying \textbf{quantum computational paradigm}, progressing from hardware-native approaches to hybrid and quantum-inspired methods:

\begin{enumerate}
    \item \textbf{Quantum Annealing Methods} (Section~\ref{sec:quantum-annealing}): We begin with Q-Seg and related approaches that exploit quantum annealing hardware (e.g., D-Wave systems). These methods reformulate segmentation as Quadratic Unconstrained Binary Optimization (QUBO) problems, leveraging quantum tunneling to escape local minima. They represent the most direct path to real quantum hardware execution today.
    
    \item \textbf{Gate-Model QAOA Methods} (Sections~\ref{sec:quantum-logismos} and \ref{sec:qcffcm}): Next, we examine algorithms based on the Quantum Approximate Optimization Algorithm (QAOA), which operates on gate-based quantum computers. QuantumLOGISMOS applies QAOA to geometric-constrained medical image segmentation, while QCFFCM integrates QAOA into fuzzy clustering for high-resolution scientific imagery.
    
    \item \textbf{Quantum-Inspired Classical Methods} (Section~\ref{sec:ffqoak}): Finally, we present FFQOAK, a quantum-inspired approach that simulates quantum dynamics on classical hardware. This represents an important category: methods that capture quantum algorithmic ideas without requiring quantum hardware.
\end{enumerate}

\begin{keyconceptbox}[Why These Algorithms?]
The selected algorithms illustrate three fundamental questions in quantum image segmentation:
\begin{itemize}
    \item \textbf{Problem formulation:} How is segmentation encoded as a quantum-compatible optimization? (QUBO, Ising Hamiltonians, variational objectives)
    \item \textbf{Hardware requirements:} What quantum resources are needed? (Annealers vs. gate-model vs. classical simulation)
    \item \textbf{Scalability trade-offs:} How do theoretical advantages translate to practical image sizes?
\end{itemize}
\end{keyconceptbox}

\subsection{Distinguishing Characteristics}

To help navigate the diversity of approaches, we highlight key differentiators:

\begin{table}[H]
\centering
\small
\begin{tabular}{@{}lccc@{}}
\toprule
\textbf{Algorithm} & \textbf{Paradigm} & \textbf{Classical Foundation} & \textbf{Primary Innovation} \\
\midrule
Q-Seg & Quantum Annealing & Graph cuts & QUBO min-cut formulation \\
QuantumLOGISMOS & QAOA (Gate-model) & LOGISMOS & Multiple optimal solutions \\
QCFFCM & QAOA (Gate-model) & Fuzzy C-Means & 3-ADMM-H hybrid framework \\
FFQOAK & Quantum-inspired & K-Means & Quantum dynamics simulation \\
\bottomrule
\end{tabular}
\caption{Overview of quantum segmentation algorithms covered in this review}
\label{tab:algo-overview}
\end{table}

Each algorithm builds upon a well-established classical segmentation technique (graph cuts, LOGISMOS, FCM, or K-means) and introduces quantum enhancements at specific computational bottlenecks. This hybrid philosophy---quantum acceleration of classical frameworks---reflects the practical reality of the NISQ era.

\begin{notebox}[Reading Guide]
Readers interested in \textbf{immediate quantum hardware deployment} should focus on Q-Seg (quantum annealing). Those exploring \textbf{medical imaging applications} will find QuantumLOGISMOS most relevant. For \textbf{large-scale practical segmentation} without quantum hardware access, FFQOAK and QCFFCM offer deployable solutions today.
\end{notebox}

The section concludes with a \textbf{comparative analysis} that synthesizes these approaches, providing guidance on algorithm selection based on application requirements, available hardware, and performance priorities.

%% Include individual algorithm files
%% Each algorithm is in its own file for parallel editing

% chapters/algorithms/quantum-annealing.tex
% Quantum Annealing-Based Image Segmentation Algorithms

\subsection{Quantum Annealing for Image Segmentation}
\label{sec:quantum-annealing}

Quantum Annealing (QA) represents a computational paradigm designed to find global minima of objective functions by exploiting quantum fluctuations~\cite{kadowaki1998quantum,li2020quantum}. Unlike classical optimization methods, QA leverages \textbf{quantum tunneling} to escape local minima by passing through energy barriers rather than climbing over them~\cite{mcgeoch2014adiabatic}. This approach has gained significant attention for image segmentation tasks~\cite{wang2022review,ruan2021qip}.

\begin{keyconceptbox}[Quantum Annealing Principle]
Quantum annealing implements a time-dependent Hamiltonian that evolves from a simple initial configuration to the problem encoding:
\begin{equation}
    H(t) = A(t) \sum_i \sigma_i^x + B(t) \left( \sum_{i<j} J_{ij} \sigma_i^z \sigma_i^z + \sum_i h_i \sigma_i^z \right)
\end{equation}
where $A(t)$ and $B(t)$ control the annealing schedule, and the problem Hamiltonian encodes the optimization objective.
\end{keyconceptbox}

\subsubsection{Relationship to Image Segmentation}

Image segmentation problems are naturally suited for quantum annealing due to their NP-hard nature~\cite{lucas2014}. The relationship is established by reformulating segmentation tasks into Quadratic Unconstrained Binary Optimization (QUBO) problems:

\begin{center}
\textbf{Image Segmentation} $\rightarrow$ \textbf{QUBO Formulation} $\rightarrow$ \textbf{Quantum Annealing}
\end{center}

The general QUBO objective function takes the form:
\begin{equation}
    \arg\min_{x} x^{T}Qx = \arg\min_{x} \sum_{i = 1}^{n}l_{i}x_{i} + \sum_{1 \leq i < j \leq n} q_{ij}x_{i}x_{j}
\end{equation}
where $x \in \{0,1\}^{n}$ is the binary variable vector encoding pixel assignments, and $Q \in \mathbb{R}^{n \times n}$ is the QUBO matrix with diagonal elements $l_i$ (linear coefficients) and off-diagonal elements $q_{ij}$ (quadratic coefficients).

\subsubsection{Q-Seg: Quantum Annealing-Based Unsupervised Image Segmentation}

Q-Seg is a quantum annealing approach that formulates image segmentation as a minimum graph cut problem~\cite{boykov2001graphcuts}. The method constructs a grid graph from the image, computes similarity weights between neighboring pixels, and solves the resulting QUBO problem on a quantum annealer~\cite{dwave2020}.

\begin{definitionbox}[Q-Seg Pipeline]
The Q-Seg algorithm consists of six main steps:
\begin{enumerate}
    \item Construct grid graph $G(V,w)$ from input image $I$
    \item Compute pixel similarity weights for edge connections
    \item Formulate minimum cut as QUBO objective
    \item Map QUBO to quantum annealer architecture (embedding)
    \item Execute quantum annealing (sampling)
    \item Decode lowest-energy sample into segmentation mask
\end{enumerate}
\end{definitionbox}

\paragraph{Step 1: Image to Graph Transformation}

A grid graph $G(V,w)$ is constructed with one node per image pixel. Edges connect neighboring pixels and represent spatial adjacency. Edge weights are computed using a Gaussian similarity metric:

\begin{equation}
    w'(p_{i},p_{j}) = 1 - \exp\left( - \frac{(I(p_{i}) - I(p_{j}))^{2}}{2\sigma^{2}} \right)
\end{equation}

where $I(p_i)$ and $I(p_j)$ are the intensity values of neighboring pixels $p_i$ and $p_j$, and $\sigma$ is a scale parameter controlling sensitivity to intensity differences.

The weights are then normalized to range $[-1, 1]$:
\begin{equation}
    w(p_{i},p_{j}) = -1 \times \left( (b - a) \cdot \frac{w'(p_{i},p_{j}) - \min(w)}{\max(w) - \min(w)} + a \right)
\end{equation}
where $a = -1$ and $b = 1$, and $\min(w), \max(w)$ denote the minimum and maximum over all raw similarities.

\paragraph{Step 2: Minimum-Cut Objective}

For an undirected weighted graph $G(V,w)$, the minimum cut cost partitions vertices into two disjoint sets $A$ and $\bar{A}$:
\begin{equation}
    \text{MINCUT}(G) = \arg\min_{A,\bar{A}} \sum_{i \in A,\, j \in \bar{A}} w(v_{i},v_{j})
\end{equation}

Using binary encoding where $x_{v_{i}} \in \{0,1\}$ represents pixel membership:
\begin{equation}
    x^{*} = \arg\min_{x} \sum_{1 \leq i < j \leq n} x_{v_{i}} (1 - x_{v_{j}}) w(v_{i},v_{j})
\end{equation}

This formulation expresses the minimum cut as an explicit quadratic polynomial in binary variables, directly convertible to the canonical QUBO matrix $Q$.

\paragraph{Step 3: Quantum Annealer Mapping}

The QUBO is mapped to the quantum annealer's physical qubit architecture through a process called \textbf{minor embedding}. Due to limited connectivity in real quantum hardware, logical problem variables may require multiple physical qubits (chains) to represent.

\begin{algorithm}[H]
\caption{Q-Seg Algorithm}
\begin{algorithmic}[1]
\Require Image $I$, similarity parameter $\sigma$
\Ensure Binary segmentation mask $M$
\State Construct grid graph $G(V,w)$ from image $I$
\State Compute edge weights using Gaussian similarity
\State Formulate QUBO matrix $Q$ for minimum cut
\State Embed QUBO onto quantum annealer topology
\State Execute quantum annealing with multiple samples
\State Select lowest-energy sample $X^{*}$
\State Decode $X^{*}$ into segmentation mask $M$
\State \Return $M$
\end{algorithmic}
\end{algorithm}

\begin{tipbox}
Q-Seg is particularly effective for binary segmentation tasks where the goal is to separate foreground from background. For multi-class segmentation, hierarchical or iterative approaches can be employed.
\end{tipbox}

\subsubsection{SAR Image Segmentation with Quantum Annealing}

Synthetic Aperture Radar (SAR) image segmentation presents unique challenges due to speckle noise and complex texture patterns. Quantum annealing combined with Markov Random Fields (MRF) provides a robust framework for SAR segmentation.

The MRF formulation encodes both data fidelity (how well pixel labels match observed intensities) and spatial coherence (preference for smooth label configurations) in the QUBO objective:

\begin{equation}
    E(L) = \sum_{i} U_i(l_i) + \sum_{(i,j) \in \mathcal{N}} V_{ij}(l_i, l_j)
\end{equation}

where $U_i(l_i)$ is the unary potential for pixel $i$ with label $l_i$, and $V_{ij}(l_i, l_j)$ is the pairwise potential encouraging neighboring pixels to have similar labels.

\begin{warningbox}
Current quantum annealers have limited qubit counts (thousands of qubits), restricting direct application to small image patches. Practical implementations often use:
\begin{itemize}
    \item Downsampling or patch-based processing
    \item Hybrid classical-quantum approaches
    \item Iterative refinement strategies
\end{itemize}
\end{warningbox}

\subsubsection{Complexity Analysis}

\begin{notebox}[Computational Considerations]
\textbf{Classical min-cut:} $O(V \cdot E)$ using efficient flow algorithms

\textbf{Quantum annealing:} 
\begin{itemize}
    \item Problem setup: $O(n^2)$ for QUBO construction
    \item Embedding: Problem-dependent, can be computationally expensive
    \item Annealing time: Typically microseconds to milliseconds per sample
    \item Sampling: Multiple runs required for solution quality
\end{itemize}

The potential quantum advantage lies in the ability to find global optima through quantum tunneling, particularly for problems with rugged energy landscapes and many local minima.
\end{notebox}

\subsubsection{Practical Considerations}

\begin{enumerate}
    \item \textbf{Problem Size Limitations:} Current D-Wave quantum annealers support $\sim$5000 qubits with sparse connectivity~\cite{dwave2020}. Image segmentation problems scale as $O(n^2)$ for $n$ pixels, limiting direct application to small images ($\sim$70×70 pixels).
    
    \item \textbf{Embedding Overhead:} Minor embedding can significantly increase the effective problem size, further constraining applicable image dimensions.
    
    \item \textbf{Noise and Errors:} Quantum annealing is susceptible to thermal noise and control errors. Multiple samples and post-processing are typically required.
    
    \item \textbf{Parameter Tuning:} The annealing schedule, chain strength (for embedding), and other parameters require careful tuning for optimal performance.
\end{enumerate}

% chapters/algorithms/quantum-logismos.tex
% QuantumLOGISMOS: QAOA-Based Geometric Constrained Image Segmentation

\subsection{QuantumLOGISMOS: QAOA for Geometric Constrained Segmentation}
\label{sec:quantum-logismos}

QuantumLOGISMOS extends the classical LOGISMOS (Layered Optimal Graph Image Segmentation of Multiple Objects and Surfaces) framework~\cite{li2004,li2006} by incorporating quantum optimization via the Quantum Approximate Optimization Algorithm (QAOA)~\cite{farhi2014}. This approach is particularly suited for medical image analysis where geometric constraints ensure anatomically realistic segmentations~\cite{le2023quantumlogismos,yan2024medical,wei2023qml}.

\begin{keyconceptbox}[QuantumLOGISMOS Core Idea]
The framework combines:
\begin{itemize}
    \item \textbf{Graph-theoretic formulation:} Surface segmentation as minimum $s$-$t$ cut
    \item \textbf{Geometric constraints:} Smoothness conditions on surface variation
    \item \textbf{QUBO encoding:} Graph structure embedded in quadratic objective
    \item \textbf{QAOA optimization:} Hybrid quantum-classical parameter search
\end{itemize}
The key advantage is the ability to find \textbf{multiple optimal solutions}, providing alternative segmentations that classical deterministic methods may miss.
\end{keyconceptbox}

\subsubsection{Classical LOGISMOS Foundation}

\paragraph{Problem Formulation}

Given an image $\mathcal{I}$ with spatial dimensions $(X, Y, Z)$, we construct a directed graph $\mathcal{G} = (\mathcal{V}, \mathcal{E})$ where each pixel (2D) or voxel (3D) corresponds to a node $v \in \mathcal{V}$, organized into columns along a specific direction.

A surface $\mathcal{S}$ is defined by a mapping function:
\begin{equation}
    s: \text{Column} \rightarrow \text{Node}
\end{equation}
where $s(x) = k$ indicates node $k$ in column $x$ belongs to the surface.

Each node $(x, k)$ has an associated cost derived from image features:
\begin{equation}
    c_s(x,k) = -\log P(\text{node is on surface} \mid \text{image features})
\end{equation}

The optimization objective minimizes total surface cost:
\begin{equation}
    \hat{s} = \arg\min_{s} \sum_{x} c_s(x, s(x))
\end{equation}

\paragraph{Terminal Weight Transformation}

Rather than directly minimizing costs, we transform to terminal weights:
\begin{equation}
    w_s(x,k) = 
    \begin{cases} 
        -1 & \text{if } k=1 \\
        c_s(x,k) - c_s(x,k-1) & \text{otherwise}
    \end{cases}
\end{equation}

For a closed set $S$ (nodes below the surface):
\begin{equation}
    W_s = \sum_{x} \sum_{k \in S_x} w_s(x,k) = \sum_{x} c_s(x, s(x)) + \text{constant}
\end{equation}

Thus minimizing $W_s$ is equivalent to minimizing the original cost function.

\paragraph{Graph Edge Constraints}

Three edge types enforce geometric constraints in the LOGISMOS graph:

\begin{definitionbox}[LOGISMOS Edge Types]
\begin{enumerate}
    \item \textbf{Intra-column edges ($\mathcal{E}_{\text{intra}}$):}
    \[
    \forall x, \forall k > 1: \text{edge } (x,k) \rightarrow (x,k-1) \text{ with capacity } \infty
    \]
    Ensures exactly one cut per column.
    
    \item \textbf{Inter-column edges ($\mathcal{E}_{\text{inter}}$):} Given smoothness parameter $\delta$:
    \[
    \forall \text{adjacent } x, x': \text{edges } (x,k) \rightarrow (x', \max(1, k-\delta)) \text{ with capacity } \infty
    \]
    Enforces $|s(x) - s(x')| \leq \delta$ (smoothness constraint).
    
    \item \textbf{Terminal edges ($\mathcal{E}_{W}$):}
    \begin{itemize}
        \item For $w_s(v) < 0$: edge $s \rightarrow v$ with capacity $|w_s(v)|$
        \item For $w_s(v) > 0$: edge $v \rightarrow t$ with capacity $|w_s(v)|$
    \end{itemize}
\end{enumerate}
\end{definitionbox}

\subsubsection{Quantum Formulation}

\paragraph{QUBO Conversion}

For each node $i$, we define a binary variable $x_i \in \{0,1\}$:
\begin{itemize}
    \item $x_i = 0 \rightarrow$ node in source set $S$
    \item $x_i = 1 \rightarrow$ node in sink set $T$
\end{itemize}

For a directed edge $i \rightarrow j$ with capacity $w_{ij}$, the cut contribution is:
\begin{equation}
    F_{(i,j)}(x_i, x_j) = w_{ij}(x_j - x_i x_j)
\end{equation}

This equals $w_{ij}$ if the edge is cut ($x_i=0, x_j=1$), and 0 otherwise.

To enforce source/sink constraints ($x_s=0, x_t=1$):
\begin{equation}
    F_{(s,t)}(x_s, x_t) = \varepsilon(x_s x_t - x_s)
\end{equation}

The complete QUBO objective becomes:
\begin{equation}
    F_C(\mathbf{x}) = \sum_{(i,j) \in \mathcal{E}} w_{ij} (x_j - x_i x_j) + \varepsilon (x_s x_t - x_s)
\end{equation}
where $\varepsilon = 1 + \sum_{(i,j) \in \mathcal{E}} w_{ij}$ ensures valid cuts have lower energy.

\paragraph{Matrix Form}

The QUBO can be written in matrix form:
\begin{equation}
    F_C(\mathbf{x}) = \mathbf{x}^T \mathbf{Q} \mathbf{x}
\end{equation}
where $\mathbf{Q}$ is symmetric with elements:
\begin{equation}
    Q_{ii} = \sum_{j: i \rightarrow j} w_{ij}, \quad Q_{ij} = -\frac{w_{ij}}{2} \text{ for } i \neq j \text{ with edge } i \rightarrow j
\end{equation}

\paragraph{Ising Hamiltonian Mapping}

Binary variables map to qubit states via the transformation:
\begin{equation}
    x_i = \frac{1 - Z_i}{2}
\end{equation}
where $Z_i$ is the Pauli-Z operator on qubit $i$.

The problem Hamiltonian becomes:
\begin{equation}
    H_C = \sum_{i,j} Q_{ij} \frac{1-Z_i}{2} \frac{1-Z_j}{2}
\end{equation}

Expanding to standard Ising form:
\begin{equation}
    H_C = \text{constant} + \sum_i h_i Z_i + \sum_{i<j} J_{ij} Z_i Z_j
\end{equation}
where:
\begin{equation}
    h_i = -\frac{1}{4} \sum_j (Q_{ij} + Q_{ji}), \quad J_{ij} = \frac{1}{4} Q_{ij} \text{ for } i \neq j
\end{equation}

The ground state energy $E_0$ of $H_C$ satisfies: $E_0 = \min_{\mathbf{x}} F_C(\mathbf{x})$.

\subsubsection{QAOA Implementation}

The Quantum Approximate Optimization Algorithm provides a hybrid quantum-classical approach to finding the ground state of $H_C$.

\paragraph{Circuit Structure}

The QAOA circuit prepares parameterized quantum states through alternating applications of problem and mixer unitaries:

\textbf{Initial state:} Uniform superposition
\begin{equation}
    |\psi_0\rangle = |+\rangle^{\otimes n} = \frac{1}{\sqrt{2^n}} \sum_{z \in \{0,1\}^n} |z\rangle
\end{equation}

\textbf{Problem unitary:}
\begin{equation}
    U_C(\gamma) = e^{-i\gamma H_C}
\end{equation}

\textbf{Mixer unitary:}
\begin{equation}
    U_M(\beta) = e^{-i\beta \sum_i X_i}
\end{equation}

For $p$ layers with parameters $\boldsymbol{\gamma} = (\gamma_1, \ldots, \gamma_p)$ and $\boldsymbol{\beta} = (\beta_1, \ldots, \beta_p)$:
\begin{equation}
    |\psi(\boldsymbol{\gamma}, \boldsymbol{\beta})\rangle = \prod_{k=1}^p U_M(\beta_k) U_C(\gamma_k) |\psi_0\rangle
\end{equation}

\paragraph{Hybrid Optimization Loop}

The energy expectation:
\begin{equation}
    E(\boldsymbol{\gamma}, \boldsymbol{\beta}) = \langle \psi(\boldsymbol{\gamma}, \boldsymbol{\beta}) | H_C | \psi(\boldsymbol{\gamma}, \boldsymbol{\beta}) \rangle
\end{equation}
is minimized using classical optimization (e.g., SPSA, COBYLA, or gradient descent).

\begin{algorithm}[H]
\caption{QuantumLOGISMOS Algorithm}
\begin{algorithmic}[1]
\Require Image $\mathcal{I}$, smoothness parameter $\delta$, QAOA depth $p$
\Ensure Optimal surface $\hat{s}$
\State \textbf{// Classical Preprocessing}
\State Compute node costs $c_s(x,k)$ from image features
\State Transform to terminal weights $w_s(x,k)$
\State Construct LOGISMOS graph with edge constraints
\State \textbf{// QUBO Formulation}
\State Build QUBO matrix $\mathbf{Q}$ from graph structure
\State Convert to Ising Hamiltonian $H_C$
\State \textbf{// QAOA Optimization}
\State Initialize parameters $\boldsymbol{\gamma}, \boldsymbol{\beta}$
\While{not converged}
    \State Prepare $|\psi(\boldsymbol{\gamma}, \boldsymbol{\beta})\rangle$ on quantum processor
    \State Measure to estimate $E(\boldsymbol{\gamma}, \boldsymbol{\beta})$
    \State Update $\boldsymbol{\gamma}, \boldsymbol{\beta}$ using classical optimizer
\EndWhile
\State \textbf{// Solution Extraction}
\State Measure final state to obtain bitstring $\mathbf{x}^*$
\State Decode $\mathbf{x}^*$ to surface nodes
\State \Return Optimal surface $\hat{s}$
\end{algorithmic}
\end{algorithm}

\subsubsection{Advantages of the Quantum Approach}

\begin{tipbox}[Key Benefits]
\begin{enumerate}
    \item \textbf{Multiple optimal solutions:} QAOA can discover multiple minima with equal or near-equal energy, providing alternative segmentations that classical deterministic methods miss.
    
    \item \textbf{Global optimization:} Quantum superposition enables simultaneous exploration of the entire solution space.
    
    \item \textbf{Geometric constraint preservation:} The QUBO formulation naturally encodes smoothness and single-cut-per-column constraints.
    
    \item \textbf{Scalability potential:} As quantum hardware improves, larger images can be processed directly.
\end{enumerate}
\end{tipbox}

\subsubsection{Complexity Analysis}

\begin{notebox}[Computational Complexity]
\textbf{Classical LOGISMOS:} $O(V \cdot E)$ using max-flow algorithms

\textbf{QuantumLOGISMOS:}
\begin{itemize}
    \item Graph construction: $O(n \cdot \delta)$ where $n$ is pixel count
    \item QUBO formulation: $O(|\mathcal{E}|)$ for edge processing
    \item QAOA circuit depth: $O(p \cdot n)$ for $p$ layers
    \item Classical optimization: Iteration-dependent
\end{itemize}

For NISQ-era devices, practical implementations are limited to small images ($\sim$50-100 nodes), but demonstrate proof-of-concept for larger-scale future applications.
\end{notebox}

\subsubsection{Applications in Medical Imaging}

QuantumLOGISMOS has been demonstrated on:
\begin{itemize}
    \item \textbf{2D synthetic images:} Validation of correctness against classical solutions
    \item \textbf{3D volumetric data:} Surface extraction in medical imaging
    \item \textbf{Multi-surface problems:} Simultaneous segmentation of multiple anatomical boundaries
\end{itemize}

\begin{warningbox}
Current limitations include:
\begin{itemize}
    \item Qubit count restricts image size
    \item Noise in NISQ devices affects solution quality
    \item Classical simulation for validation becomes intractable for large problems
    \item QAOA parameter optimization can be challenging for deep circuits
\end{itemize}
\end{warningbox}

% chapters/algorithms/quantum-ffqoak.tex
% FFQOAK: Fast Forward Quantum Optimization Algorithm with K-means

\subsection{FFQOAK: Hybrid Quantum-Inspired K-means Clustering}
\label{sec:ffqoak}

The Fast Forward Quantum Optimization Algorithm combined with K-means (FFQOAK)~\cite{singh2021ffqoak} represents a hybrid quantum-classical approach that addresses fundamental limitations of standard K-means clustering~\cite{macqueen1967kmeans,xu2015clustering}: sensitivity to initialization and convergence to local optima. By incorporating quantum-inspired optimization~\cite{jasso2023qknn}, FFQOAK achieves more robust global optimization for image segmentation tasks.

\begin{keyconceptbox}[FFQOAK Core Innovation]
FFQOAK integrates quantum-inspired optimization principles with classical K-means through a two-phase framework:
\begin{itemize}
    \item \textbf{Phase 1 (Quantum Optimization):} FFQOA searches for optimal cluster centers using quantum-inspired dynamics
    \item \textbf{Phase 2 (Classical Clustering):} K-means performs pixel assignment based on optimized centers
\end{itemize}
This synergy combines global search capability with efficient local refinement.
\end{keyconceptbox}

\subsubsection{Background: K-means Limitations}

Standard K-means clustering minimizes within-cluster variance:
\begin{equation}
    J = \sum_{i=1}^{n} \sum_{z=1}^{\theta} \|P_i - C_z\|^2
\end{equation}
where $P_i$ represents pixel intensity values, $C_z$ are cluster centroids, $\theta$ is the number of clusters, and $n$ is the total number of pixels.

The algorithm alternates between:
\begin{enumerate}
    \item \textbf{Assignment:} $\text{Label}_i = \arg\min_z \|P_i - C_z\|^2$
    \item \textbf{Update:} $C_z = \frac{1}{|S_z|} \sum_{i \in S_z} P_i$
\end{enumerate}

\begin{warningbox}[K-means Limitations]
Critical issues affecting segmentation quality:
\begin{itemize}
    \item \textbf{Initialization sensitivity:} Different starting centers produce vastly different results
    \item \textbf{Local optima:} Algorithm terminates at suboptimal solutions
    \item \textbf{Fixed cluster count:} Requires prior knowledge of $\theta$
\end{itemize}
These limitations are particularly problematic in medical imaging where accurate delineation is crucial~\cite{jun2020covid}.
\end{warningbox}

\subsubsection{The FFQOA Framework}

The Fast Forward Quantum Optimization Algorithm draws inspiration from quantum mechanical principles to achieve superior optimization performance.

\paragraph{Quantum Mechanical Inspiration}

FFQOA models optimization as a quantum system where:
\begin{itemize}
    \item Each potential solution is represented as a \textbf{quantum} ($Q_k$)
    \item The solution space forms a quantum potential landscape
    \item Quantum properties (location, movement, displacement) guide search dynamics
\end{itemize}

\paragraph{Quantum System Initialization}

The quantum system is initialized using a Schrödinger equation-inspired formulation:
\begin{equation}
    Q_k(e) = \phi \cdot Q1_k(e) + (1-\phi) \cdot Q2_k(e)
\end{equation}
where:
\begin{itemize}
    \item $Q_k(e)$ represents the $k$-th quantum at epoch $e$
    \item $\phi = \frac{1}{\sqrt{2}}(1 + i)$ is a complex superposition coefficient
    \item $Q1_k(e) = C_{min} + r_1 \cdot (C_{max} - C_{min})$
    \item $Q2_k(e) = C_{min} + r_2 \cdot (C_{max} - C_{min})$
    \item $r_1, r_2 \in [0,1]$ are random numbers
\end{itemize}

The complex coefficient $\phi$ introduces quantum superposition principles, allowing each quantum to explore multiple search regions simultaneously.

\paragraph{Quantum Properties}

\begin{definitionbox}[FFQOA Quantum Properties]
\textbf{Location} ($L_k(e)$): Quantum position in search space
\begin{equation}
    L_k(e) = \frac{1}{Q_k(e)} e^{-2/Q_k(e)}
\end{equation}

\textbf{Movement} ($M_k(e)$): Tendency to change position
\begin{equation}
    M_k(e) = \left|Q_k(e) - \frac{L_k(e)}{2} \ln(1/m_f)\right|
\end{equation}
where $m_f \in [0,1]$ is the quantum movement factor.

\textbf{Displacement} ($D_k(e)$): Actual position change
\begin{equation}
    D_k(e) = 2 \cdot |L_k(e) - M_k(e)|
\end{equation}
\end{definitionbox}

\paragraph{Three-Component Search Enhancement}

The key innovation of FFQOA is its three-component movement update mechanism:
\begin{equation}
    M_k(e+1) = M_1 + M_2 + M_3
\end{equation}

\begin{enumerate}
    \item \textbf{Preceding Movement Component} ($M_1$):
    \begin{equation}
        M_1 = w \cdot M_k(e)
    \end{equation}
    where $w$ is an inertia weight maintaining search momentum (exploitation).
    
    \item \textbf{Local Search Component} ($M_2$):
    \begin{equation}
        M_2 = c_1 \cdot r_3 \cdot (pBD_k(e) - D_k(e))
    \end{equation}
    where $pBD_k(e)$ is the personal best displacement (local learning).
    
    \item \textbf{Global Search Component} ($M_3$):
    \begin{equation}
        M_3 = c_2 \cdot r_4 \cdot (gBD(e) - D_k(e))
    \end{equation}
    where $gBD(e)$ is the global best displacement (collective knowledge).
\end{enumerate}

The displacement update follows:
\begin{equation}
    D_k(e+1) = D_k(e) + M_k(e+1)
\end{equation}

\begin{tipbox}[Exploration-Exploitation Balance]
The three components create a balanced search strategy:
\begin{itemize}
    \item $M_1$ maintains momentum (exploitation of current trajectory)
    \item $M_2$ incorporates individual experience (local refinement)
    \item $M_3$ leverages collective information (global exploration)
\end{itemize}
\end{tipbox}

\subsubsection{FFQOAK Integration}

\paragraph{Mathematical Integration}

Cluster centers are represented as quantum displacements:
\begin{itemize}
    \item Each set of potential centers forms a quantum displacement $D_j(e)$
    \item K-means objective evaluates fitness: $\hat{J}_j(e) = J(D_j(e))$
    \item FFQOA optimizes $D_j(e)$ to minimize $\hat{J}_j(e)$
\end{itemize}

\begin{algorithm}[H]
\caption{FFQOAK Algorithm}
\begin{algorithmic}[1]
\Require Image $I_p$ with gray levels $G_{ld} = \{P_1, P_2, \ldots, P_n\}$
\Require Number of clusters $\theta$, quantum population $N_Q$, max epochs $E$
\Ensure Optimal cluster centers, segmented image $I_s$
\State \textbf{// Initialization}
\State Initialize $N_Q$ quanta with random cluster centers
\State Compute initial fitness $\hat{J}_j(0)$ for each quantum
\State Set $pBD_j(0) = D_j(0)$ and identify $gBD(0)$
\For{$e = 1$ to $E$}
    \For{each quantum $j = 1$ to $N_Q$}
        \State \textbf{// Update quantum properties}
        \State Update location $L_j(e)$
        \State Compute three-component movement $M_j(e)$
        \State Update displacement $D_j(e) = D_j(e-1) + M_j(e)$
        \State \textbf{// K-means evaluation}
        \State Decode $D_j(e)$ to cluster centers $C_1, \ldots, C_\theta$
        \State Assign pixels to nearest centers
        \State Compute fitness $\hat{J}_j(e)$
        \State \textbf{// Update best solutions}
        \If{$\hat{J}_j(e) < \hat{J}(pBD_j)$}
            \State $pBD_j(e) = D_j(e)$
        \EndIf
    \EndFor
    \State Update global best $gBD(e)$ if improved
\EndFor
\State Extract optimal centers from $gBD(E)$
\State Generate segmented image $I_s$
\State \Return Cluster centers, $I_s$
\end{algorithmic}
\end{algorithm}

\subsubsection{Parameter Configuration}

\begin{notebox}[Recommended Parameters]
Based on empirical studies for medical image segmentation:
\begin{itemize}
    \item Number of clusters: $\theta = 3$ (for COVID-19 CT segmentation)
    \item Quantum population: $N_Q = 30$
    \item Inertia weight: $w = 0.7$
    \item Learning coefficients: $c_1 = c_2 = 1.5$
    \item Maximum epochs: $E = 100$
    \item Movement factor: $m_f = 0.5$
\end{itemize}
\end{notebox}

\subsubsection{Advantages Over Classical Methods}

\paragraph{Escaping Local Optima}

The quantum tunneling effect, modeled through complex number formulation and stochastic components, enables FFQOAK to escape poor local minima that trap standard K-means.

\paragraph{Initialization Independence}

Unlike classical K-means:
\begin{itemize}
    \item Multiple quanta explore different initial configurations simultaneously
    \item Global best tracking ensures convergence to high-quality solutions
    \item Reduced sensitivity to random initialization
\end{itemize}

\paragraph{Computational Advantages}

\begin{itemize}
    \item \textbf{Parallel exploration:} Quantum system inherently explores multiple solutions
    \item \textbf{Adaptive search:} Balance between exploration and exploitation evolves
    \item \textbf{Information sharing:} Global best component enables rapid propagation of good solutions
\end{itemize}


\subsubsection{Complexity Analysis}

\textbf{Time Complexity:}
\begin{itemize}
    \item Per epoch: $O(N_Q \cdot n \cdot \theta)$ for fitness evaluation
    \item Total: $O(E \cdot N_Q \cdot n \cdot \theta)$
\end{itemize}

\textbf{Space Complexity:} $O(N_Q \cdot \theta + n)$ for storing quanta and pixel data

Compared to standard K-means $O(I \cdot n \cdot \theta)$ where $I$ is iteration count, FFQOAK introduces overhead from quantum population but achieves better solution quality with fewer restarts.

% chapters/algorithms/quantum-fuzzy-qcffcm.tex
% QCFFCM: Quantum Computing-based Fast Fuzzy C-Means

\subsection{QCFFCM: Quantum-Enhanced Fast Fuzzy C-Means Clustering}
\label{sec:qcffcm}

The Quantum Computing-based Fast Fuzzy C-Means (QCFFCM) algorithm~\cite{illarionov2020qcffcm} integrates quantum optimization into the classical Fast Fuzzy C-Means (FFCM) framework~\cite{bezdek1984fcm} using the Quantum Approximate Optimization Algorithm (QAOA)~\cite{farhi2014} and a three-block ADMM heuristic (3-ADMM-H)~\cite{boyd2011admm}. This hybrid approach achieves near-real-time performance for large-scale image segmentation while maintaining competitive accuracy.

\begin{keyconceptbox}[QCFFCM Innovation]
QCFFCM addresses the computational bottleneck of fuzzy clustering on high-resolution images by:
\begin{itemize}
    \item Histogram-based reduction of search space (from millions of pixels to 256 intensity levels)
    \item QUBO subproblem formulation for quantum optimization
    \item 3-ADMM-H framework for hybrid quantum-classical iteration
    \item QAOA execution for global optimization of membership functions
\end{itemize}
\textbf{Result:} $\sim$12 seconds per 4096×4096 image (3-20× faster than classical methods)
\end{keyconceptbox}

\subsubsection{Classical Fuzzy C-Means Foundation}

\paragraph{Traditional FCM Objective}

Fuzzy C-Means minimizes the weighted within-cluster variance:
\begin{equation}
    J_E = \sum_{j=1}^{N} \sum_{i=1}^{c} \mu_{ij}^m \| x_j - v_i \|^2
\end{equation}
where:
\begin{itemize}
    \item $N$: number of pixels (e.g., $\approx$16.8 million for 4096×4096)
    \item $c$: number of clusters
    \item $\mu_{ij} \in [0,1]$: fuzzy membership of pixel $j$ in cluster $i$
    \item $m$: fuzziness exponent (typically $m = 2$)
    \item $x_j$: intensity of pixel $j$
    \item $v_i$: centroid of cluster $i$
\end{itemize}

The iterative updates are:
\begin{equation}
    v_i = \frac{\sum_{j=1}^{N} \mu_{ij}^m x_j}{\sum_{j=1}^{N} \mu_{ij}^m}
\end{equation}
\begin{equation}
    \mu_{ij} = \left[ \sum_{k=1}^{c} \left( \frac{\| x_j - v_i \|}{\| x_j - v_k \|} \right)^{\frac{2}{m-1}} \right]^{-1}
\end{equation}

\paragraph{Histogram-Based Acceleration (FFCM)}

To reduce computational complexity from $O(N)$ to $O(256)$, FFCM operates on intensity histograms:

\begin{definitionbox}[FFCM Histogram Formulation]
Define intensity occurrence frequency:
\begin{equation}
    n_L = \sum_{j=0}^{N} \delta[x_j - L], \quad L \in \{0,1,\ldots,255\}
\end{equation}
where $\delta[x_j - L] = 1$ if $x_j = L$, else 0.

The FFCM energy function becomes:
\begin{equation}
    J_E = \sum_{L=0}^{255} \sum_{i=1}^{c} \mu_{iL}^m n_L \| L - v_i \|^2
\end{equation}

Modified update equations:
\begin{equation}
    v_i = \frac{\sum_{L=0}^{255} \mu_{iL}^m n_L \cdot L}{\sum_{L=0}^{255} \mu_{iL}^m n_L}
\end{equation}
\begin{equation}
    \mu_{iL} = \left[ \sum_{k=1}^{c} \left( \frac{\| L - v_i \|}{\| L - v_k \|} \right)^{\frac{2}{m-1}} \right]^{-1}
\end{equation}
\end{definitionbox}

This reduces the search space from $N$ pixels to 256 intensity levels, enabling efficient processing of high-resolution images.

\subsubsection{Quantum Optimization via 3-ADMM-H}

\paragraph{Constrained Optimization Formulation}

The FFCM energy minimization is reformulated as a constrained optimization problem:
\begin{equation}
    \begin{aligned}
        \min_{\mu} & \quad \sum_{j=1}^{N} \sum_{i=1}^{c} \mu_{ij}^m \| x_j - v_i \|^2 \\
        \text{subject to:} & \quad J_E \geq 1 \\
        & \quad v_i \geq 0, \quad i = 1, \ldots, c \\
        & \quad \sum_{i=1}^{c} v_i \geq 1
    \end{aligned}
\end{equation}

\paragraph{Three-Block ADMM Heuristic}

To handle inequality constraints, an auxiliary variable $z$ is introduced:
\begin{equation}
    p^* = \inf \left\{ J_E \;\middle|\; J_E - z = 1, v \in \Upsilon, z \in \mathbb{R}_+^n \right\}
\end{equation}

The scaled augmented Lagrangian becomes:
\begin{equation}
    \begin{aligned}
        L_\rho(\mu, z, y, v, \lambda, \eta) &= J_E + \rho \sum_{i,j} \lambda_{i,j} (J_E - z - 1) \\
        &+ \frac{\rho}{2} \sum_{i,j} (J_E - z - 1)^2 \\
        &+ \tau \rho \eta \left( \sum_{i=1}^{c} v_i - y - 1 \right) \\
        &+ \frac{\tau \rho}{2} \left( \sum_{i=1}^{c} v_i - y - 1 \right)^2
    \end{aligned}
\end{equation}
where $\rho, \tau$ are penalty parameters, and $\lambda, \eta$ are Lagrange multipliers.

\paragraph{Block Decomposition}

The 3-ADMM-H splits the optimization into three subproblems:

\begin{enumerate}
    \item \textbf{QUBO Subproblem} (solved via QAOA on quantum device)
    \item \textbf{Convex Optimization Subproblem} (classical update of $\mu_{ij}$)
    \item \textbf{Convex-Quadratic Subproblem} (classical update of $v_i$)
\end{enumerate}

\subsubsection{QAOA Implementation for QCFFCM}

\paragraph{Quantum State Preparation}

The QAOA circuit prepares parameterized states:
\begin{equation}
    |\gamma, \beta\rangle = U(B, \beta_P) U(C, \gamma_P) \cdots U(B, \beta_1) U(C, \gamma_1) | s \rangle
\end{equation}

\textbf{Problem Hamiltonian:}
\begin{equation}
    U(C, \gamma) = e^{-i\gamma C} = \prod_{\alpha=1}^{M} e^{-i\gamma C_\alpha}
\end{equation}

\textbf{Mixing Hamiltonian:}
\begin{equation}
    U(B, \beta) = e^{-i\beta B} = \prod_{t=1}^{k} e^{-i\beta \sigma_t^X}
\end{equation}
where $B = \sum_{t=1}^{k} \sigma_t^X$.

\textbf{Initial state} (uniform superposition):
\begin{equation}
    | s \rangle = \frac{1}{\sqrt{2^k}} \sum_{\mathfrak{z}} | \mathfrak{z} \rangle
\end{equation}

Parameter ranges: $\gamma \in [0, 2\pi]$, $\beta \in [0, \pi]$.

\paragraph{Multi-Start Strategy}

A multi-start approach is employed to escape local optima in the parameter optimization landscape.

\begin{algorithm}[H]
\caption{QCFFCM Algorithm}
\begin{algorithmic}[1]
\Require Image $I$ (4096×4096), number of clusters $c$, convergence threshold $\epsilon$
\Ensure Segmentation mask, cluster memberships
\State \textbf{// Preprocessing}
\State Compute histogram $\{n_L\}_{L=0}^{255}$ from image $I$
\State Initialize centroids $v_i$ and memberships $\mu_{iL}$
\State Initialize Lagrange multipliers $\lambda, \eta$
\Repeat
    \State \textbf{// Block 1: QAOA (Quantum)}
    \State Formulate QUBO from current Lagrangian
    \State Prepare QAOA circuit with parameters $\gamma, \beta$
    \State Execute QAOA to optimize QUBO subproblem
    \State Extract solution for energy contribution
    \State \textbf{// Block 2: Membership Update (Classical)}
    \State Update memberships $\mu_{iL}$ via convex optimization
    \State \textbf{// Block 3: Centroid Update (Classical)}
    \State Update centroids $v_i$ via quadratic programming
    \State \textbf{// Lagrange Multiplier Update}
    \State Update $\lambda, \eta$ based on constraint violations
\Until{$\| \hat{\mu}^{k+1} - \hat{\mu}^k \| < \epsilon$}
\State Apply post-processing (CHT, morphological operations)
\State \Return Segmentation mask
\end{algorithmic}
\end{algorithm}

\subsubsection{Post-Processing for Refinement}

For applications like solar coronal hole detection, additional post-processing steps refine the segmentation:

\begin{enumerate}
    \item \textbf{Circular Hough Transform (CHT):} Initializes contours to remove solar limbs and isolate disk interior
    
    \item \textbf{Classification Rules:}
    \begin{itemize}
        \item Foreground outside contour $\rightarrow$ background
        \item Background inside contour $\rightarrow$ background
        \item Foreground inside contour $\rightarrow$ target candidate
    \end{itemize}
    
    \item \textbf{Area-based Morphological Operations:} Remove regions with area below threshold $T_A$
\end{enumerate}

\subsubsection{Experimental Results}

\paragraph{Application: Solar Coronal Hole Detection}

Dataset: 365 SDO/AIA 193Å images (4096×4096 resolution) from 2017.

\begin{notebox}[Performance Metrics]
\textbf{F1 Score:}
\begin{equation}
    \text{F1} = \frac{2|A \cap B|}{|A| + |B|}
\end{equation}
where $A$ = detected region, $B$ = ground truth.

\textbf{Results:}
\begin{itemize}
    \item Peak F1 score: \textbf{0.9199} (February 2017)
    \item Annual average: \textbf{0.75--0.85}
    \item Execution time: \textbf{$\sim$12 seconds} per image
\end{itemize}
\end{notebox}

\paragraph{Comparative Performance}

\begin{center}
\begin{tabular}{@{}lcc@{}}
\toprule
\textbf{Method} & \textbf{Execution Time} & \textbf{F1 Score} \\
\midrule
Classical FCM & 40--60 seconds & 0.70--0.80 \\
CNN-based & 60--240 seconds & 0.75--0.85 \\
CHIMERA & 30--50 seconds & 0.75--0.82 \\
SPoCA & 25--40 seconds & 0.73--0.80 \\
\textbf{QCFFCM} & \textbf{$\sim$12 seconds} & \textbf{0.75--0.92} \\
\bottomrule
\end{tabular}
\end{center}

\begin{tipbox}[Key Achievements]
\begin{itemize}
    \item \textbf{3-20× speedup} over classical methods
    \item \textbf{Competitive accuracy} with state-of-the-art approaches
    \item \textbf{Near-real-time processing} for high-resolution images
    \item Better capture of boundary curvatures compared to CNN and classical methods
\end{itemize}
\end{tipbox}

\subsubsection{Complexity Analysis}

\textbf{Classical FCM:} $O(I \cdot N \cdot c)$ where $I$ = iterations, $N$ = pixels

\textbf{FFCM (histogram-based):} $O(I \cdot 256 \cdot c)$ - independent of image size

\textbf{QCFFCM:}
\begin{itemize}
    \item Histogram computation: $O(N)$ (one-time)
    \item QAOA per iteration: $O(p \cdot k)$ for $p$ layers, $k$ qubits
    \item Classical updates: $O(256 \cdot c)$
    \item Total per iteration: $O(p \cdot k + 256 \cdot c)$
\end{itemize}

The quantum advantage emerges from the QAOA's ability to explore the solution space more efficiently, particularly for problems with complex energy landscapes.

\subsubsection{Practical Considerations}

\begin{warningbox}[Implementation Notes]
\begin{itemize}
    \item Current implementations use \textbf{classical QAOA simulation}
    \item System requirements are modest (tested on Intel Core i5, 4GB RAM)
    \item \textbf{Threshold selection} ($T_A$) for morphological operations remains manual
    \item Performance gains increase with quantum hardware improvements
\end{itemize}
\end{warningbox}

\subsubsection{Future Directions}

\begin{enumerate}
    \item \textbf{True quantum execution:} Deployment on actual quantum hardware for further speedup
    \item \textbf{Adaptive thresholding:} Automated selection of morphological parameters
    \item \textbf{Multi-class extension:} Generalization beyond binary segmentation
    \item \textbf{Real-time systems:} Integration into operational space weather monitoring
\end{enumerate}

% chapters/algorithms/comparative-analysis.tex
% Comparative Analysis of Quantum Clustering Algorithms

\subsection{Comparative Analysis}

Table \ref{tab:quantum-comparison} summarizes the key characteristics of quantum clustering algorithms for image segmentation.

\begin{table}[h]
\centering
\caption{Comparison of Quantum Clustering Algorithms}
\label{tab:quantum-comparison}
\begin{tabular}{@{}lccc@{}}
\toprule
\textbf{Algorithm} & \textbf{Speedup} & \textbf{Hardware} & \textbf{Maturity} \\
\midrule
q-Means & Exponential & Fault-tolerant + qRAM & Theoretical \\
Quantum Spectral & Exponential & Fault-tolerant & Theoretical \\
QAOA Clustering & Potential & NISQ & Experimental \\
VQE Clustering & Heuristic & NISQ & Experimental \\
Quantum Fuzzy & Quadratic & Fault-tolerant & Theoretical \\
%% Add new algorithms here following the same format:
%% Algorithm Name & Speedup Type & Hardware Req & Maturity Level \\
\bottomrule
\end{tabular}
\end{table}

\begin{keyconceptbox}[Current State of the Art]
While theoretical quantum algorithms promise significant speedups, practical quantum image segmentation remains limited by:
\begin{itemize}
    \item Data loading bottleneck: Encoding classical images into quantum states
    \item Hardware limitations: Qubit counts, error rates, and coherence times
    \item Scalability: Current demonstrations limited to small images
\end{itemize}
Variational methods on NISQ devices represent the most promising near-term approach.
\end{keyconceptbox}


%% ==============================================
%% HOW TO ADD A NEW ALGORITHM:
%% ==============================================
%% 1. Create a new file in chapters/algorithms/ (e.g., quantum-newmethod.tex)
%% 2. Start the file with \subsection{Your Algorithm Name}
%% 3. Add your content (subsubsections, equations, boxes, etc.)
%% 4. Add an \input line above this comment block:
%%    \input{chapters/algorithms/quantum-newmethod}
%% 5. Update the comparative analysis table in comparative-analysis.tex
%% ==============================================

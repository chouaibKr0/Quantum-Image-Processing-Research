% chapters/chapitre0.tex
% Quantum Optimization Method for Geometric Constrained Image Segmentation
% Based on work by Nam H. Le, Milan Sonka, Fatima Toor
% Department of Electrical and Computer Engineering, University of Iowa

\chapter{Quantum Optimization for Geometric Constrained Image Segmentation}
\label{ch:quantum-optimization}

\begin{chapterintro}
This chapter explores quantum optimization methods for geometric constrained image segmentation, combining graph-theoretic approaches with hybrid quantum-classical optimization. We present the QuantumLOGISMOS framework which leverages the Quantum Approximate Optimization Algorithm (QAOA) to solve surface segmentation problems.
\end{chapterintro}

\section{Overview}
\label{sec:quantum-opt-overview}

Quantum image processing is an emerging field attracting attention from both quantum computing and image processing communities. We propose a novel method combining a graph-theoretic approach for optimal surface segmentation with hybrid quantum-classical optimization of the problem-directed graph. Surface segmentation is modeled classically as a graph partitioning problem with smoothness constraints to control surface variation for realistic segmentation. The problem-specific graph characteristics are embedded in a quadratic objective function whose minimum corresponds to the ground state energy of an equivalent Ising Hamiltonian. 

This work explores the use of quantum processors in image segmentation problems, with important applications in medical image analysis. We present a theoretical basis for the quantum implementation of LOGISMOS and simulation results on simple images using the Quantum Approximate Optimization Algorithm (QAOA). The proposed approach can solve geometric-constrained surface segmentation problems optimally with the capability of locating multiple minimum points corresponding to globally minimal solutions.

\begin{keyconceptbox}
\textbf{Key Concepts:} quantum computing, quantum algorithm, combinatorial optimization, image segmentation, graph theory, QAOA, LOGISMOS framework
\end{keyconceptbox}

\section{Introduction}
\label{sec:introduction}

Image segmentation, particularly in medical imaging, involves partitioning an image into meaningful anatomical regions. The LOGISMOS framework (Layered Optimal Graph Image Segmentation of Multiple Objects and Surfaces) \cite{li2004, li2006, wu2002} reformulates surface segmentation as finding optimal boundaries subject to geometric constraints. Traditional algorithms solve this via maximum flow/minimum cut on directed graphs using classical optimization methods, but may miss alternative optimal solutions due to their deterministic nature.

Quantum computers offer potential advantages for combinatorial optimization problems through superposition (evaluating multiple solutions simultaneously), entanglement (correlated exploration of solution spaces), and quantum tunneling (escaping local minima more effectively). The Quantum Approximate Optimization Algorithm (QAOA) \cite{farhi2014} provides a hybrid quantum-classical approach to approximate solutions of NP-hard problems by encoding them as ground state problems of Ising Hamiltonians.

This work introduces QuantumLOGISMOS, which: (1) maps LOGISMOS graph constraints to Quadratic Unconstrained Binary Optimization (QUBO) formulation, (2) implements quantum optimization via QAOA with classical parameter tuning, (3) demonstrates the method on synthetic 2D and 3D images, and (4) shows quantum advantage in finding multiple optimal segmentation solutions.

\section{Classical LOGISMOS Framework}
\label{sec:classical-logismos}

\subsection{Mathematical Formulation}
Given an image $\mathcal{I}$ with spatial dimensions $(X, Y, Z)$, we represent it as a directed graph $\mathcal{G} = (\mathcal{V}, \mathcal{E})$ where each pixel (2D) or voxel (3D) corresponds to a node $v \in \mathcal{V}$, organized into columns along a specific direction.

A surface $\mathcal{S}$ is defined by a function:
\[
s: \text{Column} \rightarrow \text{Node}
\]
where $s(x) = k$ indicates node $k$ in column $x$ belongs to the surface.

Each node $(x, k)$ has an associated cost:
\[
c_s(x,k) = -\log P(\text{node is on surface} \mid \text{image features})
\]
Lower cost indicates higher likelihood of being on the desired surface.

The optimization objective is to find the surface minimizing total cost:
\[
\hat{s} = \arg\min_{s} \sum_{x} c_s(x, s(x))
\]

\subsection{Graph Construction}
Instead of directly minimizing costs, we transform to terminal weights:
\[
w_s(x,k) = 
\begin{cases} 
-1 & \text{if } k=1 \\
c_s(x,k) - c_s(x,k-1) & \text{otherwise}
\end{cases}
\]

For a closed set $S$ (nodes below the surface):
\[
W_s = \sum_{x} \sum_{k \in S_x} w_s(x,k) = \sum_{x} c_s(x, s(x)) + \text{constant}
\]
Thus minimizing $W_s$ is equivalent to minimizing the original cost function.

\subsection{Edge Constraints}
Three edge types enforce geometric constraints:

\begin{enumerate}
    \item \textbf{Intra-column edges ($\mathcal{E}_{\text{intra}}$)}: 
    $\forall x, \forall k > 1: \text{edge } (x,k) \rightarrow (x,k-1) \text{ with capacity } \infty$
    
    Ensures exactly one cut per column.
    
    \item \textbf{Inter-column edges ($\mathcal{E}_{\text{inter}}$)}: 
    Given smoothness parameter $\delta$:
    $\forall \text{adjacent } x, x': \text{edges } (x,k) \rightarrow (x', \max(1, k-\delta)) \text{ with capacity } \infty$
    
    Enforces $|s(x) - s(x')| \leq \delta$.
    
    \item \textbf{Terminal edges ($\mathcal{E}_{W}$)}: 
    For nodes with $w_s(v) < 0$: edge $s \rightarrow v$ with capacity $|w_s(v)|$
    
    For nodes with $w_s(v) > 0$: edge $v \rightarrow t$ with capacity $|w_s(v)|$
\end{enumerate}

\subsection{Minimum Cut Reformulation}
The optimal surface corresponds to the minimum $s$-$t$ cut partitioning nodes into source set $S$ (containing $s$) and sink set $T$ (containing $t$). The cut capacity equals the total terminal weight of the corresponding closed set.

\section{QuantumLOGISMOS Framework}
\label{sec:quantum-logismos-framework}

\begin{figure}[h]
    \centering
    \includegraphics[width=0.9\textwidth]{figures/quantum-logismos-framework.png}
    \caption{The proposed Quantum LOGISMOS framework: (1) Estimate cost functions and calculate terminal weights, (2) Introduce internal connections within columns to ensure that the optimal surface cut passes through each column only once, (3) Add inter-edges to impose smoothness condition, (4) Add source and sink nodes, add problem-specific edges, (5) Assign qubit to graph node, (6) QAOA optimization, (7) Bitstring solution and minimum closed set found.}
    \label{fig:quantum-logismos-framework}
\end{figure}

The QuantumLOGISMOS framework, illustrated in Figure~\ref{fig:quantum-logismos-framework}, integrates classical graph-theoretic segmentation with quantum optimization through seven systematic steps:

\begin{enumerate}
    \item \textbf{Cost Function Estimation and Terminal Weight Calculation}: Based on the input image, compute node costs reflecting surface likelihood and transform them to terminal weights using Equation~\ref{eq:terminal-weights}.
    
    \item \textbf{Intra-column Edge Construction}: Create infinite-capacity directed edges within each column to ensure exactly one cut per column, enforcing the single-surface constraint.
    
    \item \textbf{Inter-column Edge Addition}: Connect adjacent columns with infinite-capacity edges to impose smoothness constraints, restricting surface variation between neighboring columns.
    
    \item \textbf{Source/Sink Node Addition}: Introduce source ($s$) and sink ($t$) nodes with capacity-weighted connections to all graph nodes based on terminal weight signs.
    
    \item \textbf{Qubit Assignment}: Map each graph node to a qubit, where the qubit state ($|0\rangle$ or $|1\rangle$) represents node assignment to source or sink sets respectively.
    
    \item \textbf{QAOA Optimization}: Execute the Quantum Approximate Optimization Algorithm to find the minimum cut by minimizing the problem Hamiltonian derived from the graph structure.
    
    \item \textbf{Solution Extraction}: Decode the optimal bitstring obtained from QAOA measurements to identify the minimum closed set and corresponding optimal surface.
\end{enumerate}

This hybrid framework leverages classical preprocessing (steps 1-4) to encode geometric constraints, while utilizing quantum optimization (steps 5-7) to explore the solution space more comprehensively than classical approaches.

\section{Quantum Formulation}
\label{sec:quantum-formulation}

\subsection{QUBO Conversion}
For each node $i$, define binary variable $x_i \in \{0,1\}$ where:
\begin{itemize}
    \item $x_i = 0$ $\rightarrow$ node in source set $S$
    \item $x_i = 1$ $\rightarrow$ node in sink set $T$
\end{itemize}

For directed edge $i \rightarrow j$ with capacity $w_{ij}$:
\[
F_{(i,j)}(x_i, x_j) = x_j - x_i x_j
\]
This equals $w_{ij}$ if the edge is cut ($x_i=0, x_j=1$), and 0 otherwise.

To enforce $x_s=0, x_t=1$:
\[
F_{(s,t)}(x_s, x_t) = x_s x_t - x_s
\]

Complete QUBO objective:
\[
F_C(\mathbf{x}) = \sum_{(i,j) \in \mathcal{E}} w_{ij} (x_j - x_i x_j) + \varepsilon (x_s x_t - x_s)
\]
where $\varepsilon = 1 + \sum_{(i,j) \in \mathcal{E}} w_{ij}$ ensures valid cuts have lower energy.

\subsection{Matrix Formulation}
\[
F_C(\mathbf{x}) = \mathbf{x}^T \mathbf{Q} \mathbf{x}
\]
where $\mathbf{Q}$ is symmetric with:
\[
Q_{ii} = \sum_{j: i \rightarrow j} w_{ij}, \quad Q_{ij} = -\frac{w_{ij}}{2} \ \text{for} \ i \neq j \ \text{with edge} \ i \rightarrow j
\]

\subsection{Ising Hamiltonian Mapping}
Binary variables map to qubit states via:
\[
x_i = \frac{1 - Z_i}{2}
\]
where $Z_i$ is the Pauli-Z operator on qubit $i$.

Problem Hamiltonian:
\[
H_C = \sum_{i,j} Q_{ij} \frac{1-Z_i}{2} \frac{1-Z_j}{2}
\]

Expanding to standard Ising form:
\[
H_C = \text{constant} + \sum_i h_i Z_i + \sum_{i<j} J_{ij} Z_i Z_j
\]
where:
\[
h_i = -\frac{1}{4} \sum_j (Q_{ij} + Q_{ji}), \quad J_{ij} = \frac{1}{4} Q_{ij} \ (i \neq j)
\]

Ground state energy $E_0$ of $H_C$ satisfies:
\[
E_0 = \min_{\mathbf{x}} F_C(\mathbf{x})
\]

\section{Quantum Optimization via QAOA}
\label{sec:qaoa}

\subsection{QAOA Circuit Structure}
Initial state: uniform superposition
\[
|\psi_0\rangle = |+\rangle^{\otimes n} = \frac{1}{\sqrt{2^n}} \sum_{z \in \{0,1\}^n} |z\rangle
\]

Problem unitary: $U_C(\gamma) = e^{-i\gamma H_C}$

Mixer unitary: $U_M(\beta) = e^{-i\beta \sum_i X_i}$

For $p$ layers with parameters $\gamma = (\gamma_1, ..., \gamma_p)$, $\beta = (\beta_1, ..., \beta_p)$:
\[
|\psi(\gamma, \beta)\rangle = \prod_{k=1}^p U_M(\beta_k) U_C(\gamma_k) |\psi_0\rangle
\]

\subsection{Hybrid Optimization}
Energy estimation:
\[
E(\gamma, \beta) = \langle \psi(\gamma, \beta) | H_C | \psi(\gamma, \beta) \rangle
\]

Classical optimization uses Simultaneous Perturbation Stochastic Approximation (SPSA) \cite{spall1998}:
\begin{algorithm}[H]
\caption{Hybrid Quantum-Classical Optimization}
\begin{algorithmic}[1]
\State Initialize parameters $\gamma, \beta$
\While{not converged}
    \State Prepare $|\psi(\gamma, \beta)\rangle$ on quantum processor/simulator
    \State Measure to estimate $E(\gamma, \beta)$
    \State Compute gradients via finite differences
    \State Update $\gamma, \beta$ using SPSA to minimize $E$
\EndWhile
\State Measure final state to obtain optimal bitstring $\mathbf{x}^*$
\State Decode $\mathbf{x}^*$ to surface nodes
\end{algorithmic}
\end{algorithm}
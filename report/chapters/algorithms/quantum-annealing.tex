% chapters/algorithms/quantum-annealing.tex
% Quantum Annealing-Based Image Segmentation Algorithms

\subsection{Quantum Annealing for Image Segmentation}
\label{sec:quantum-annealing}

Quantum Annealing (QA) represents a computational paradigm designed to find global minima of objective functions by exploiting quantum fluctuations~\cite{kadowaki1998quantum,li2020quantum}. Unlike classical optimization methods, QA leverages \textbf{quantum tunneling} to escape local minima by passing through energy barriers rather than climbing over them~\cite{mcgeoch2014adiabatic}. This approach has gained significant attention for image segmentation tasks~\cite{wang2022review,ruan2021qip}.

\begin{keyconceptbox}[Quantum Annealing Principle]
Quantum annealing implements a time-dependent Hamiltonian that evolves from a simple initial configuration to the problem encoding:
\begin{equation}
    H(t) = A(t) \sum_i \sigma_i^x + B(t) \left( \sum_{i<j} J_{ij} \sigma_i^z \sigma_i^z + \sum_i h_i \sigma_i^z \right)
\end{equation}
where $A(t)$ and $B(t)$ control the annealing schedule, and the problem Hamiltonian encodes the optimization objective.
\end{keyconceptbox}

\subsubsection{Relationship to Image Segmentation}

Image segmentation problems are naturally suited for quantum annealing due to their NP-hard nature~\cite{lucas2014}. The relationship is established by reformulating segmentation tasks into Quadratic Unconstrained Binary Optimization (QUBO) problems:

\begin{center}
\textbf{Image Segmentation} $\rightarrow$ \textbf{QUBO Formulation} $\rightarrow$ \textbf{Quantum Annealing}
\end{center}

The general QUBO objective function takes the form:
\begin{equation}
    \arg\min_{x} x^{T}Qx = \arg\min_{x} \sum_{i = 1}^{n}l_{i}x_{i} + \sum_{1 \leq i < j \leq n} q_{ij}x_{i}x_{j}
\end{equation}
where $x \in \{0,1\}^{n}$ is the binary variable vector encoding pixel assignments, and $Q \in \mathbb{R}^{n \times n}$ is the QUBO matrix with diagonal elements $l_i$ (linear coefficients) and off-diagonal elements $q_{ij}$ (quadratic coefficients).

\subsubsection{Q-Seg: Quantum Annealing-Based Unsupervised Image Segmentation}

Q-Seg is a quantum annealing approach that formulates image segmentation as a minimum graph cut problem~\cite{boykov2001graphcuts}. The method constructs a grid graph from the image, computes similarity weights between neighboring pixels, and solves the resulting QUBO problem on a quantum annealer~\cite{dwave2020}.

\begin{definitionbox}[Q-Seg Pipeline]
The Q-Seg algorithm consists of six main steps:
\begin{enumerate}
    \item Construct grid graph $G(V,w)$ from input image $I$
    \item Compute pixel similarity weights for edge connections
    \item Formulate minimum cut as QUBO objective
    \item Map QUBO to quantum annealer architecture (embedding)
    \item Execute quantum annealing (sampling)
    \item Decode lowest-energy sample into segmentation mask
\end{enumerate}
\end{definitionbox}

\paragraph{Step 1: Image to Graph Transformation}

A grid graph $G(V,w)$ is constructed with one node per image pixel. Edges connect neighboring pixels and represent spatial adjacency. Edge weights are computed using a Gaussian similarity metric:

\begin{equation}
    w'(p_{i},p_{j}) = 1 - \exp\left( - \frac{(I(p_{i}) - I(p_{j}))^{2}}{2\sigma^{2}} \right)
\end{equation}

where $I(p_i)$ and $I(p_j)$ are the intensity values of neighboring pixels $p_i$ and $p_j$, and $\sigma$ is a scale parameter controlling sensitivity to intensity differences.

The weights are then normalized to range $[-1, 1]$:
\begin{equation}
    w(p_{i},p_{j}) = -1 \times \left( (b - a) \cdot \frac{w'(p_{i},p_{j}) - \min(w)}{\max(w) - \min(w)} + a \right)
\end{equation}
where $a = -1$ and $b = 1$, and $\min(w), \max(w)$ denote the minimum and maximum over all raw similarities.

\paragraph{Step 2: Minimum-Cut Objective}

For an undirected weighted graph $G(V,w)$, the minimum cut cost partitions vertices into two disjoint sets $A$ and $\bar{A}$:
\begin{equation}
    \text{MINCUT}(G) = \arg\min_{A,\bar{A}} \sum_{i \in A,\, j \in \bar{A}} w(v_{i},v_{j})
\end{equation}

Using binary encoding where $x_{v_{i}} \in \{0,1\}$ represents pixel membership:
\begin{equation}
    x^{*} = \arg\min_{x} \sum_{1 \leq i < j \leq n} x_{v_{i}} (1 - x_{v_{j}}) w(v_{i},v_{j})
\end{equation}

This formulation expresses the minimum cut as an explicit quadratic polynomial in binary variables, directly convertible to the canonical QUBO matrix $Q$.

\paragraph{Step 3: Quantum Annealer Mapping}

The QUBO is mapped to the quantum annealer's physical qubit architecture through a process called \textbf{minor embedding}. Due to limited connectivity in real quantum hardware, logical problem variables may require multiple physical qubits (chains) to represent.

\begin{algorithm}[H]
\caption{Q-Seg Algorithm}
\begin{algorithmic}[1]
\Require Image $I$, similarity parameter $\sigma$
\Ensure Binary segmentation mask $M$
\State Construct grid graph $G(V,w)$ from image $I$
\State Compute edge weights using Gaussian similarity
\State Formulate QUBO matrix $Q$ for minimum cut
\State Embed QUBO onto quantum annealer topology
\State Execute quantum annealing with multiple samples
\State Select lowest-energy sample $X^{*}$
\State Decode $X^{*}$ into segmentation mask $M$
\State \Return $M$
\end{algorithmic}
\end{algorithm}

\begin{tipbox}
Q-Seg is particularly effective for binary segmentation tasks where the goal is to separate foreground from background. For multi-class segmentation, hierarchical or iterative approaches can be employed.
\end{tipbox}

\subsubsection{SAR Image Segmentation with Quantum Annealing}

Synthetic Aperture Radar (SAR) image segmentation presents unique challenges due to speckle noise and complex texture patterns. Quantum annealing combined with Markov Random Fields (MRF) provides a robust framework for SAR segmentation.

The MRF formulation encodes both data fidelity (how well pixel labels match observed intensities) and spatial coherence (preference for smooth label configurations) in the QUBO objective:

\begin{equation}
    E(L) = \sum_{i} U_i(l_i) + \sum_{(i,j) \in \mathcal{N}} V_{ij}(l_i, l_j)
\end{equation}

where $U_i(l_i)$ is the unary potential for pixel $i$ with label $l_i$, and $V_{ij}(l_i, l_j)$ is the pairwise potential encouraging neighboring pixels to have similar labels.

\begin{warningbox}
Current quantum annealers have limited qubit counts (thousands of qubits), restricting direct application to small image patches. Practical implementations often use:
\begin{itemize}
    \item Downsampling or patch-based processing
    \item Hybrid classical-quantum approaches
    \item Iterative refinement strategies
\end{itemize}
\end{warningbox}

\subsubsection{Complexity Analysis}

\begin{notebox}[Computational Considerations]
\textbf{Classical min-cut:} $O(V \cdot E)$ using efficient flow algorithms

\textbf{Quantum annealing:} 
\begin{itemize}
    \item Problem setup: $O(n^2)$ for QUBO construction
    \item Embedding: Problem-dependent, can be computationally expensive
    \item Annealing time: Typically microseconds to milliseconds per sample
    \item Sampling: Multiple runs required for solution quality
\end{itemize}

The potential quantum advantage lies in the ability to find global optima through quantum tunneling, particularly for problems with rugged energy landscapes and many local minima.
\end{notebox}

\subsubsection{Practical Considerations}

\begin{enumerate}
    \item \textbf{Problem Size Limitations:} Current D-Wave quantum annealers support $\sim$5000 qubits with sparse connectivity~\cite{dwave2020}. Image segmentation problems scale as $O(n^2)$ for $n$ pixels, limiting direct application to small images ($\sim$70×70 pixels).
    
    \item \textbf{Embedding Overhead:} Minor embedding can significantly increase the effective problem size, further constraining applicable image dimensions.
    
    \item \textbf{Noise and Errors:} Quantum annealing is susceptible to thermal noise and control errors. Multiple samples and post-processing are typically required.
    
    \item \textbf{Parameter Tuning:} The annealing schedule, chain strength (for embedding), and other parameters require careful tuning for optimal performance.
\end{enumerate}

% chapters/algorithms/quantum-logismos.tex
% QuantumLOGISMOS: QAOA-Based Geometric Constrained Image Segmentation

\subsection{QuantumLOGISMOS: QAOA for Geometric Constrained Segmentation}
\label{sec:quantum-logismos}

QuantumLOGISMOS extends the classical LOGISMOS (Layered Optimal Graph Image Segmentation of Multiple Objects and Surfaces) framework~\cite{li2004,li2006} by incorporating quantum optimization via the Quantum Approximate Optimization Algorithm (QAOA)~\cite{farhi2014}. This approach is particularly suited for medical image analysis where geometric constraints ensure anatomically realistic segmentations~\cite{le2023quantumlogismos,yan2024medical,wei2023qml}.

\begin{keyconceptbox}[QuantumLOGISMOS Core Idea]
The framework combines:
\begin{itemize}
    \item \textbf{Graph-theoretic formulation:} Surface segmentation as minimum $s$-$t$ cut
    \item \textbf{Geometric constraints:} Smoothness conditions on surface variation
    \item \textbf{QUBO encoding:} Graph structure embedded in quadratic objective
    \item \textbf{QAOA optimization:} Hybrid quantum-classical parameter search
\end{itemize}
The key advantage is the ability to find \textbf{multiple optimal solutions}, providing alternative segmentations that classical deterministic methods may miss.
\end{keyconceptbox}

\subsubsection{Classical LOGISMOS Foundation}

\paragraph{Problem Formulation}

Given an image $\mathcal{I}$ with spatial dimensions $(X, Y, Z)$, we construct a directed graph $\mathcal{G} = (\mathcal{V}, \mathcal{E})$ where each pixel (2D) or voxel (3D) corresponds to a node $v \in \mathcal{V}$, organized into columns along a specific direction.

A surface $\mathcal{S}$ is defined by a mapping function:
\begin{equation}
    s: \text{Column} \rightarrow \text{Node}
\end{equation}
where $s(x) = k$ indicates node $k$ in column $x$ belongs to the surface.

Each node $(x, k)$ has an associated cost derived from image features:
\begin{equation}
    c_s(x,k) = -\log P(\text{node is on surface} \mid \text{image features})
\end{equation}

The optimization objective minimizes total surface cost:
\begin{equation}
    \hat{s} = \arg\min_{s} \sum_{x} c_s(x, s(x))
\end{equation}

\paragraph{Terminal Weight Transformation}

Rather than directly minimizing costs, we transform to terminal weights:
\begin{equation}
    w_s(x,k) = 
    \begin{cases} 
        -1 & \text{if } k=1 \\
        c_s(x,k) - c_s(x,k-1) & \text{otherwise}
    \end{cases}
\end{equation}

For a closed set $S$ (nodes below the surface):
\begin{equation}
    W_s = \sum_{x} \sum_{k \in S_x} w_s(x,k) = \sum_{x} c_s(x, s(x)) + \text{constant}
\end{equation}

Thus minimizing $W_s$ is equivalent to minimizing the original cost function.

\paragraph{Graph Edge Constraints}

Three edge types enforce geometric constraints in the LOGISMOS graph:

\begin{definitionbox}[LOGISMOS Edge Types]
\begin{enumerate}
    \item \textbf{Intra-column edges ($\mathcal{E}_{\text{intra}}$):}
    \[
    \forall x, \forall k > 1: \text{edge } (x,k) \rightarrow (x,k-1) \text{ with capacity } \infty
    \]
    Ensures exactly one cut per column.
    
    \item \textbf{Inter-column edges ($\mathcal{E}_{\text{inter}}$):} Given smoothness parameter $\delta$:
    \[
    \forall \text{adjacent } x, x': \text{edges } (x,k) \rightarrow (x', \max(1, k-\delta)) \text{ with capacity } \infty
    \]
    Enforces $|s(x) - s(x')| \leq \delta$ (smoothness constraint).
    
    \item \textbf{Terminal edges ($\mathcal{E}_{W}$):}
    \begin{itemize}
        \item For $w_s(v) < 0$: edge $s \rightarrow v$ with capacity $|w_s(v)|$
        \item For $w_s(v) > 0$: edge $v \rightarrow t$ with capacity $|w_s(v)|$
    \end{itemize}
\end{enumerate}
\end{definitionbox}

\subsubsection{Quantum Formulation}

\paragraph{QUBO Conversion}

For each node $i$, we define a binary variable $x_i \in \{0,1\}$:
\begin{itemize}
    \item $x_i = 0 \rightarrow$ node in source set $S$
    \item $x_i = 1 \rightarrow$ node in sink set $T$
\end{itemize}

For a directed edge $i \rightarrow j$ with capacity $w_{ij}$, the cut contribution is:
\begin{equation}
    F_{(i,j)}(x_i, x_j) = w_{ij}(x_j - x_i x_j)
\end{equation}

This equals $w_{ij}$ if the edge is cut ($x_i=0, x_j=1$), and 0 otherwise.

To enforce source/sink constraints ($x_s=0, x_t=1$):
\begin{equation}
    F_{(s,t)}(x_s, x_t) = \varepsilon(x_s x_t - x_s)
\end{equation}

The complete QUBO objective becomes:
\begin{equation}
    F_C(\mathbf{x}) = \sum_{(i,j) \in \mathcal{E}} w_{ij} (x_j - x_i x_j) + \varepsilon (x_s x_t - x_s)
\end{equation}
where $\varepsilon = 1 + \sum_{(i,j) \in \mathcal{E}} w_{ij}$ ensures valid cuts have lower energy.

\paragraph{Matrix Form}

The QUBO can be written in matrix form:
\begin{equation}
    F_C(\mathbf{x}) = \mathbf{x}^T \mathbf{Q} \mathbf{x}
\end{equation}
where $\mathbf{Q}$ is symmetric with elements:
\begin{equation}
    Q_{ii} = \sum_{j: i \rightarrow j} w_{ij}, \quad Q_{ij} = -\frac{w_{ij}}{2} \text{ for } i \neq j \text{ with edge } i \rightarrow j
\end{equation}

\paragraph{Ising Hamiltonian Mapping}

Binary variables map to qubit states via the transformation:
\begin{equation}
    x_i = \frac{1 - Z_i}{2}
\end{equation}
where $Z_i$ is the Pauli-Z operator on qubit $i$.

The problem Hamiltonian becomes:
\begin{equation}
    H_C = \sum_{i,j} Q_{ij} \frac{1-Z_i}{2} \frac{1-Z_j}{2}
\end{equation}

Expanding to standard Ising form:
\begin{equation}
    H_C = \text{constant} + \sum_i h_i Z_i + \sum_{i<j} J_{ij} Z_i Z_j
\end{equation}
where:
\begin{equation}
    h_i = -\frac{1}{4} \sum_j (Q_{ij} + Q_{ji}), \quad J_{ij} = \frac{1}{4} Q_{ij} \text{ for } i \neq j
\end{equation}

The ground state energy $E_0$ of $H_C$ satisfies: $E_0 = \min_{\mathbf{x}} F_C(\mathbf{x})$.

\subsubsection{QAOA Implementation}

The Quantum Approximate Optimization Algorithm provides a hybrid quantum-classical approach to finding the ground state of $H_C$.

\paragraph{Circuit Structure}

The QAOA circuit prepares parameterized quantum states through alternating applications of problem and mixer unitaries:

\textbf{Initial state:} Uniform superposition
\begin{equation}
    |\psi_0\rangle = |+\rangle^{\otimes n} = \frac{1}{\sqrt{2^n}} \sum_{z \in \{0,1\}^n} |z\rangle
\end{equation}

\textbf{Problem unitary:}
\begin{equation}
    U_C(\gamma) = e^{-i\gamma H_C}
\end{equation}

\textbf{Mixer unitary:}
\begin{equation}
    U_M(\beta) = e^{-i\beta \sum_i X_i}
\end{equation}

For $p$ layers with parameters $\boldsymbol{\gamma} = (\gamma_1, \ldots, \gamma_p)$ and $\boldsymbol{\beta} = (\beta_1, \ldots, \beta_p)$:
\begin{equation}
    |\psi(\boldsymbol{\gamma}, \boldsymbol{\beta})\rangle = \prod_{k=1}^p U_M(\beta_k) U_C(\gamma_k) |\psi_0\rangle
\end{equation}

\paragraph{Hybrid Optimization Loop}

The energy expectation:
\begin{equation}
    E(\boldsymbol{\gamma}, \boldsymbol{\beta}) = \langle \psi(\boldsymbol{\gamma}, \boldsymbol{\beta}) | H_C | \psi(\boldsymbol{\gamma}, \boldsymbol{\beta}) \rangle
\end{equation}
is minimized using classical optimization (e.g., SPSA, COBYLA, or gradient descent).

\begin{algorithm}[H]
\caption{QuantumLOGISMOS Algorithm}
\begin{algorithmic}[1]
\Require Image $\mathcal{I}$, smoothness parameter $\delta$, QAOA depth $p$
\Ensure Optimal surface $\hat{s}$
\State \textbf{// Classical Preprocessing}
\State Compute node costs $c_s(x,k)$ from image features
\State Transform to terminal weights $w_s(x,k)$
\State Construct LOGISMOS graph with edge constraints
\State \textbf{// QUBO Formulation}
\State Build QUBO matrix $\mathbf{Q}$ from graph structure
\State Convert to Ising Hamiltonian $H_C$
\State \textbf{// QAOA Optimization}
\State Initialize parameters $\boldsymbol{\gamma}, \boldsymbol{\beta}$
\While{not converged}
    \State Prepare $|\psi(\boldsymbol{\gamma}, \boldsymbol{\beta})\rangle$ on quantum processor
    \State Measure to estimate $E(\boldsymbol{\gamma}, \boldsymbol{\beta})$
    \State Update $\boldsymbol{\gamma}, \boldsymbol{\beta}$ using classical optimizer
\EndWhile
\State \textbf{// Solution Extraction}
\State Measure final state to obtain bitstring $\mathbf{x}^*$
\State Decode $\mathbf{x}^*$ to surface nodes
\State \Return Optimal surface $\hat{s}$
\end{algorithmic}
\end{algorithm}

\subsubsection{Advantages of the Quantum Approach}

\begin{tipbox}[Key Benefits]
\begin{enumerate}
    \item \textbf{Multiple optimal solutions:} QAOA can discover multiple minima with equal or near-equal energy, providing alternative segmentations that classical deterministic methods miss.
    
    \item \textbf{Global optimization:} Quantum superposition enables simultaneous exploration of the entire solution space.
    
    \item \textbf{Geometric constraint preservation:} The QUBO formulation naturally encodes smoothness and single-cut-per-column constraints.
    
    \item \textbf{Scalability potential:} As quantum hardware improves, larger images can be processed directly.
\end{enumerate}
\end{tipbox}

\subsubsection{Complexity Analysis}

\begin{notebox}[Computational Complexity]
\textbf{Classical LOGISMOS:} $O(V \cdot E)$ using max-flow algorithms

\textbf{QuantumLOGISMOS:}
\begin{itemize}
    \item Graph construction: $O(n \cdot \delta)$ where $n$ is pixel count
    \item QUBO formulation: $O(|\mathcal{E}|)$ for edge processing
    \item QAOA circuit depth: $O(p \cdot n)$ for $p$ layers
    \item Classical optimization: Iteration-dependent
\end{itemize}

For NISQ-era devices, practical implementations are limited to small images ($\sim$50-100 nodes), but demonstrate proof-of-concept for larger-scale future applications.
\end{notebox}

\subsubsection{Applications in Medical Imaging}

QuantumLOGISMOS has been demonstrated on:
\begin{itemize}
    \item \textbf{2D synthetic images:} Validation of correctness against classical solutions
    \item \textbf{3D volumetric data:} Surface extraction in medical imaging
    \item \textbf{Multi-surface problems:} Simultaneous segmentation of multiple anatomical boundaries
\end{itemize}

\begin{warningbox}
Current limitations include:
\begin{itemize}
    \item Qubit count restricts image size
    \item Noise in NISQ devices affects solution quality
    \item Classical simulation for validation becomes intractable for large problems
    \item QAOA parameter optimization can be challenging for deep circuits
\end{itemize}
\end{warningbox}

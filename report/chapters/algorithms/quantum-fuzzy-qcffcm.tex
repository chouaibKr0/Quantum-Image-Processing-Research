% chapters/algorithms/quantum-fuzzy-qcffcm.tex
% QCFFCM: Quantum Computing-based Fast Fuzzy C-Means

\subsection{QCFFCM: Quantum-Enhanced Fast Fuzzy C-Means Clustering}
\label{sec:qcffcm}

The Quantum Computing-based Fast Fuzzy C-Means (QCFFCM) algorithm~\cite{illarionov2020qcffcm} integrates quantum optimization into the classical Fast Fuzzy C-Means (FFCM) framework~\cite{bezdek1984fcm} using the Quantum Approximate Optimization Algorithm (QAOA)~\cite{farhi2014} and a three-block ADMM heuristic (3-ADMM-H)~\cite{boyd2011admm}. This hybrid approach achieves near-real-time performance for large-scale image segmentation while maintaining competitive accuracy.

\begin{keyconceptbox}[QCFFCM Innovation]
QCFFCM addresses the computational bottleneck of fuzzy clustering on high-resolution images by:
\begin{itemize}
    \item Histogram-based reduction of search space (from millions of pixels to 256 intensity levels)
    \item QUBO subproblem formulation for quantum optimization
    \item 3-ADMM-H framework for hybrid quantum-classical iteration
    \item QAOA execution for global optimization of membership functions
\end{itemize}
\textbf{Result:} $\sim$12 seconds per 4096×4096 image (3-20× faster than classical methods)
\end{keyconceptbox}

\subsubsection{Classical Fuzzy C-Means Foundation}

\paragraph{Traditional FCM Objective}

Fuzzy C-Means minimizes the weighted within-cluster variance:
\begin{equation}
    J_E = \sum_{j=1}^{N} \sum_{i=1}^{c} \mu_{ij}^m \| x_j - v_i \|^2
\end{equation}
where:
\begin{itemize}
    \item $N$: number of pixels (e.g., $\approx$16.8 million for 4096×4096)
    \item $c$: number of clusters
    \item $\mu_{ij} \in [0,1]$: fuzzy membership of pixel $j$ in cluster $i$
    \item $m$: fuzziness exponent (typically $m = 2$)
    \item $x_j$: intensity of pixel $j$
    \item $v_i$: centroid of cluster $i$
\end{itemize}

The iterative updates are:
\begin{equation}
    v_i = \frac{\sum_{j=1}^{N} \mu_{ij}^m x_j}{\sum_{j=1}^{N} \mu_{ij}^m}
\end{equation}
\begin{equation}
    \mu_{ij} = \left[ \sum_{k=1}^{c} \left( \frac{\| x_j - v_i \|}{\| x_j - v_k \|} \right)^{\frac{2}{m-1}} \right]^{-1}
\end{equation}

\paragraph{Histogram-Based Acceleration (FFCM)}

To reduce computational complexity from $O(N)$ to $O(256)$, FFCM operates on intensity histograms:

\begin{definitionbox}[FFCM Histogram Formulation]
Define intensity occurrence frequency:
\begin{equation}
    n_L = \sum_{j=0}^{N} \delta[x_j - L], \quad L \in \{0,1,\ldots,255\}
\end{equation}
where $\delta[x_j - L] = 1$ if $x_j = L$, else 0.

The FFCM energy function becomes:
\begin{equation}
    J_E = \sum_{L=0}^{255} \sum_{i=1}^{c} \mu_{iL}^m n_L \| L - v_i \|^2
\end{equation}

Modified update equations:
\begin{equation}
    v_i = \frac{\sum_{L=0}^{255} \mu_{iL}^m n_L \cdot L}{\sum_{L=0}^{255} \mu_{iL}^m n_L}
\end{equation}
\begin{equation}
    \mu_{iL} = \left[ \sum_{k=1}^{c} \left( \frac{\| L - v_i \|}{\| L - v_k \|} \right)^{\frac{2}{m-1}} \right]^{-1}
\end{equation}
\end{definitionbox}

This reduces the search space from $N$ pixels to 256 intensity levels, enabling efficient processing of high-resolution images.

\subsubsection{Quantum Optimization via 3-ADMM-H}

\paragraph{Constrained Optimization Formulation}

The FFCM energy minimization is reformulated as a constrained optimization problem:
\begin{equation}
    \begin{aligned}
        \min_{\mu} & \quad \sum_{j=1}^{N} \sum_{i=1}^{c} \mu_{ij}^m \| x_j - v_i \|^2 \\
        \text{subject to:} & \quad J_E \geq 1 \\
        & \quad v_i \geq 0, \quad i = 1, \ldots, c \\
        & \quad \sum_{i=1}^{c} v_i \geq 1
    \end{aligned}
\end{equation}

\paragraph{Three-Block ADMM Heuristic}

To handle inequality constraints, an auxiliary variable $z$ is introduced:
\begin{equation}
    p^* = \inf \left\{ J_E \;\middle|\; J_E - z = 1, v \in \Upsilon, z \in \mathbb{R}_+^n \right\}
\end{equation}

The scaled augmented Lagrangian becomes:
\begin{equation}
    \begin{aligned}
        L_\rho(\mu, z, y, v, \lambda, \eta) &= J_E + \rho \sum_{i,j} \lambda_{i,j} (J_E - z - 1) \\
        &+ \frac{\rho}{2} \sum_{i,j} (J_E - z - 1)^2 \\
        &+ \tau \rho \eta \left( \sum_{i=1}^{c} v_i - y - 1 \right) \\
        &+ \frac{\tau \rho}{2} \left( \sum_{i=1}^{c} v_i - y - 1 \right)^2
    \end{aligned}
\end{equation}
where $\rho, \tau$ are penalty parameters, and $\lambda, \eta$ are Lagrange multipliers.

\paragraph{Block Decomposition}

The 3-ADMM-H splits the optimization into three subproblems:

\begin{enumerate}
    \item \textbf{QUBO Subproblem} (solved via QAOA on quantum device)
    \item \textbf{Convex Optimization Subproblem} (classical update of $\mu_{ij}$)
    \item \textbf{Convex-Quadratic Subproblem} (classical update of $v_i$)
\end{enumerate}

\subsubsection{QAOA Implementation for QCFFCM}

\paragraph{Quantum State Preparation}

The QAOA circuit prepares parameterized states:
\begin{equation}
    |\gamma, \beta\rangle = U(B, \beta_P) U(C, \gamma_P) \cdots U(B, \beta_1) U(C, \gamma_1) | s \rangle
\end{equation}

\textbf{Problem Hamiltonian:}
\begin{equation}
    U(C, \gamma) = e^{-i\gamma C} = \prod_{\alpha=1}^{M} e^{-i\gamma C_\alpha}
\end{equation}

\textbf{Mixing Hamiltonian:}
\begin{equation}
    U(B, \beta) = e^{-i\beta B} = \prod_{t=1}^{k} e^{-i\beta \sigma_t^X}
\end{equation}
where $B = \sum_{t=1}^{k} \sigma_t^X$.

\textbf{Initial state} (uniform superposition):
\begin{equation}
    | s \rangle = \frac{1}{\sqrt{2^k}} \sum_{\mathfrak{z}} | \mathfrak{z} \rangle
\end{equation}

Parameter ranges: $\gamma \in [0, 2\pi]$, $\beta \in [0, \pi]$.

\paragraph{Multi-Start Strategy}

A multi-start approach is employed to escape local optima in the parameter optimization landscape.

\begin{algorithm}[H]
\caption{QCFFCM Algorithm}
\begin{algorithmic}[1]
\Require Image $I$ (4096×4096), number of clusters $c$, convergence threshold $\epsilon$
\Ensure Segmentation mask, cluster memberships
\State \textbf{// Preprocessing}
\State Compute histogram $\{n_L\}_{L=0}^{255}$ from image $I$
\State Initialize centroids $v_i$ and memberships $\mu_{iL}$
\State Initialize Lagrange multipliers $\lambda, \eta$
\Repeat
    \State \textbf{// Block 1: QAOA (Quantum)}
    \State Formulate QUBO from current Lagrangian
    \State Prepare QAOA circuit with parameters $\gamma, \beta$
    \State Execute QAOA to optimize QUBO subproblem
    \State Extract solution for energy contribution
    \State \textbf{// Block 2: Membership Update (Classical)}
    \State Update memberships $\mu_{iL}$ via convex optimization
    \State \textbf{// Block 3: Centroid Update (Classical)}
    \State Update centroids $v_i$ via quadratic programming
    \State \textbf{// Lagrange Multiplier Update}
    \State Update $\lambda, \eta$ based on constraint violations
\Until{$\| \hat{\mu}^{k+1} - \hat{\mu}^k \| < \epsilon$}
\State Apply post-processing (CHT, morphological operations)
\State \Return Segmentation mask
\end{algorithmic}
\end{algorithm}

\subsubsection{Post-Processing for Refinement}

For applications like solar coronal hole detection, additional post-processing steps refine the segmentation:

\begin{enumerate}
    \item \textbf{Circular Hough Transform (CHT):} Initializes contours to remove solar limbs and isolate disk interior
    
    \item \textbf{Classification Rules:}
    \begin{itemize}
        \item Foreground outside contour $\rightarrow$ background
        \item Background inside contour $\rightarrow$ background
        \item Foreground inside contour $\rightarrow$ target candidate
    \end{itemize}
    
    \item \textbf{Area-based Morphological Operations:} Remove regions with area below threshold $T_A$
\end{enumerate}

\subsubsection{Experimental Results}

\paragraph{Application: Solar Coronal Hole Detection}

Dataset: 365 SDO/AIA 193Å images (4096×4096 resolution) from 2017.

\begin{notebox}[Performance Metrics]
\textbf{F1 Score:}
\begin{equation}
    \text{F1} = \frac{2|A \cap B|}{|A| + |B|}
\end{equation}
where $A$ = detected region, $B$ = ground truth.

\textbf{Results:}
\begin{itemize}
    \item Peak F1 score: \textbf{0.9199} (February 2017)
    \item Annual average: \textbf{0.75--0.85}
    \item Execution time: \textbf{$\sim$12 seconds} per image
\end{itemize}
\end{notebox}

\paragraph{Comparative Performance}

\begin{center}
\begin{tabular}{@{}lcc@{}}
\toprule
\textbf{Method} & \textbf{Execution Time} & \textbf{F1 Score} \\
\midrule
Classical FCM & 40--60 seconds & 0.70--0.80 \\
CNN-based & 60--240 seconds & 0.75--0.85 \\
CHIMERA & 30--50 seconds & 0.75--0.82 \\
SPoCA & 25--40 seconds & 0.73--0.80 \\
\textbf{QCFFCM} & \textbf{$\sim$12 seconds} & \textbf{0.75--0.92} \\
\bottomrule
\end{tabular}
\end{center}

\begin{tipbox}[Key Achievements]
\begin{itemize}
    \item \textbf{3-20× speedup} over classical methods
    \item \textbf{Competitive accuracy} with state-of-the-art approaches
    \item \textbf{Near-real-time processing} for high-resolution images
    \item Better capture of boundary curvatures compared to CNN and classical methods
\end{itemize}
\end{tipbox}

\subsubsection{Complexity Analysis}

\textbf{Classical FCM:} $O(I \cdot N \cdot c)$ where $I$ = iterations, $N$ = pixels

\textbf{FFCM (histogram-based):} $O(I \cdot 256 \cdot c)$ - independent of image size

\textbf{QCFFCM:}
\begin{itemize}
    \item Histogram computation: $O(N)$ (one-time)
    \item QAOA per iteration: $O(p \cdot k)$ for $p$ layers, $k$ qubits
    \item Classical updates: $O(256 \cdot c)$
    \item Total per iteration: $O(p \cdot k + 256 \cdot c)$
\end{itemize}

The quantum advantage emerges from the QAOA's ability to explore the solution space more efficiently, particularly for problems with complex energy landscapes.

\subsubsection{Practical Considerations}

\begin{warningbox}[Implementation Notes]
\begin{itemize}
    \item Current implementations use \textbf{classical QAOA simulation}
    \item System requirements are modest (tested on Intel Core i5, 4GB RAM)
    \item \textbf{Threshold selection} ($T_A$) for morphological operations remains manual
    \item Performance gains increase with quantum hardware improvements
\end{itemize}
\end{warningbox}

\subsubsection{Future Directions}

\begin{enumerate}
    \item \textbf{True quantum execution:} Deployment on actual quantum hardware for further speedup
    \item \textbf{Adaptive thresholding:} Automated selection of morphological parameters
    \item \textbf{Multi-class extension:} Generalization beyond binary segmentation
    \item \textbf{Real-time systems:} Integration into operational space weather monitoring
\end{enumerate}

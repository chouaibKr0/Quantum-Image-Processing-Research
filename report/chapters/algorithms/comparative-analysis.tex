% chapters/algorithms/comparative-analysis.tex
% Comparative Analysis of Quantum Image Segmentation Algorithms

\subsection{Comparative Analysis of Quantum Segmentation Approaches}

This section synthesizes the quantum clustering algorithms presented in this chapter, comparing their theoretical foundations, computational requirements, practical applicability, and performance characteristics for image segmentation tasks~\cite{wang2022review,aburaed2017advances,cai2018survey}.

\subsubsection{Taxonomy of Quantum Segmentation Methods}

The quantum image segmentation landscape can be organized along two primary dimensions: the \textbf{quantum computational paradigm} employed and the \textbf{classical clustering foundation} extended~\cite{aimeur2007clustering,maier2005quantum}.

\begin{definitionbox}[Classification Framework]
\textbf{By Quantum Paradigm:}
\begin{itemize}
    \item \textbf{Quantum Annealing:} Q-Seg, SAR-QA (minimize QUBO via adiabatic evolution)
    \item \textbf{Gate-Model QAOA:} QuantumLOGISMOS, QCFFCM (variational hybrid optimization)
    \item \textbf{Quantum-Inspired:} FFQOAK (classical simulation of quantum dynamics)
\end{itemize}

\textbf{By Classical Foundation~\cite{xu2015clustering}:}
\begin{itemize}
    \item \textbf{Graph-Cut Methods:} Q-Seg, QuantumLOGISMOS
    \item \textbf{Centroid-Based:} FFQOAK (K-means), QCFFCM (Fuzzy C-Means)
\end{itemize}
\end{definitionbox}

\subsubsection{Algorithm Comparison Matrix}

\begin{table}[H]
\centering
\small
\begin{tabular}{@{}p{2.5cm}p{2.2cm}p{2.2cm}p{2.2cm}p{2.2cm}@{}}
\toprule
\textbf{Aspect} & \textbf{Q-Seg} & \textbf{QuantumLOGISMOS} & \textbf{FFQOAK} & \textbf{QCFFCM} \\
\midrule
\textbf{Quantum Paradigm} & Annealing & Gate-model (QAOA) & Quantum-inspired & Gate-model (QAOA) \\
\textbf{Problem Encoding} & QUBO (min-cut) & QUBO (min-cut) & None (simulation) & QUBO (3-ADMM) \\
\textbf{Hardware} & D-Wave & NISQ/Simulator & Classical only & NISQ/Simulator \\
\textbf{Segmentation Type} & Binary & Surface/Multi-surface & Multi-class & Binary/Multi-class \\
\textbf{Image Scale} & Small ($<$100px) & Small ($<$100 nodes) & Large (millions px) & Large (4096×4096) \\
\textbf{Primary Application} & General & Medical imaging & Medical (CT) & Solar imaging \\
\bottomrule
\end{tabular}
\caption{Fundamental characteristics of quantum segmentation algorithms}
\label{tab:quantum-comparison-basics}
\end{table}

\subsubsection{Theoretical Speedup Analysis}

\begin{table}[H]
\centering
\small
\begin{tabular}{@{}lcccc@{}}
\toprule
\textbf{Algorithm} & \textbf{Classical Baseline} & \textbf{Claimed Speedup} & \textbf{Speedup Type} & \textbf{Validation} \\
\midrule
Q-Seg & $O(V \cdot E)$ max-flow & Potential & Heuristic & Limited \\
QuantumLOGISMOS & $O(V \cdot E)$ max-flow & Multiple solutions & Solution quality & Simulation \\
FFQOAK & $O(I \cdot n \cdot k)$ K-means & 37\% MSE reduction & Quality improvement & Experimental \\
QCFFCM & $O(I \cdot N \cdot c)$ FCM & 3--20× time & Wall-clock time & Experimental \\
\bottomrule
\end{tabular}
\caption{Speedup claims and validation status}
\label{tab:speedup-analysis}
\end{table}

\begin{notebox}[Interpreting Speedups]
Quantum speedup claims require careful interpretation:
\begin{itemize}
    \item \textbf{Theoretical speedup:} Asymptotic complexity improvement (often requires fault-tolerant hardware)
    \item \textbf{Heuristic speedup:} Better solutions for fixed runtime (common in NISQ era)
    \item \textbf{Wall-clock speedup:} Actual measured time improvement (includes all overheads)
\end{itemize}
Most current results demonstrate heuristic or quality improvements rather than asymptotic speedup.
\end{notebox}

\subsubsection{Hardware Requirements and Scalability}

\begin{table}[H]
\centering
\small
\begin{tabular}{@{}lcccl@{}}
\toprule
\textbf{Algorithm} & \textbf{Qubits Needed} & \textbf{Gate Depth} & \textbf{Connectivity} & \textbf{Current Feasibility} \\
\midrule
Q-Seg & $n$ (pixels) & N/A (annealing) & Chimera/Pegasus & Small images only \\
QuantumLOGISMOS & $n$ (nodes) & $O(p \cdot n)$ & All-to-all ideal & Small images only \\
FFQOAK & 0 (classical) & N/A & N/A & Fully feasible \\
QCFFCM & $O(\log c)$ & $O(p \cdot c)$ & Low & Feasible (simulated) \\
\bottomrule
\end{tabular}
\caption{Hardware requirements for quantum segmentation algorithms}
\label{tab:hardware-requirements}
\end{table}

\begin{warningbox}[Scalability Bottleneck]
The primary challenge for true quantum image segmentation is the \textbf{qubit-to-pixel mapping}:
\begin{itemize}
    \item A 256×256 image requires 65,536 qubits (direct encoding)
    \item Current quantum computers: $\sim$100-1000 usable qubits
    \item Practical workarounds: downsampling, patch-based processing, hierarchical methods
\end{itemize}
\end{warningbox}

\subsubsection{Performance Benchmarks}

\paragraph{Segmentation Quality Metrics}

\begin{table}[H]
\centering
\small
\begin{tabular}{@{}lccccc@{}}
\toprule
\textbf{Algorithm} & \textbf{Dataset} & \textbf{MSE} & \textbf{PSNR (dB)} & \textbf{Jaccard} & \textbf{F1 Score} \\
\midrule
Classical K-means & COVID-19 CT & 5293.23 & 11.02 & 0.37 & -- \\
FFQOAK & COVID-19 CT & \textbf{712.30} & \textbf{19.61} & \textbf{0.90} & -- \\
QCFFCM & SDO/AIA Solar & -- & -- & -- & \textbf{0.92} \\
\bottomrule
\end{tabular}
\caption{Reported segmentation quality metrics}
\label{tab:quality-metrics}
\end{table}

\paragraph{Execution Time Comparison}

\begin{table}[H]
\centering
\small
\begin{tabular}{@{}lcccc@{}}
\toprule
\textbf{Method} & \textbf{Image Size} & \textbf{Time} & \textbf{vs. Classical} & \textbf{Notes} \\
\midrule
Classical FCM & 4096×4096 & 40--60s & Baseline & CPU only \\
CNN-based & 4096×4096 & 60--240s & 1.5--4× slower & GPU required \\
QCFFCM & 4096×4096 & $\sim$12s & \textbf{3--5× faster} & Quantum simulation \\
\bottomrule
\end{tabular}
\caption{Execution time comparison for high-resolution images}
\label{tab:execution-time}
\end{table}

\subsubsection{Strengths and Limitations Summary}

\begin{table}[H]
\centering
\small
\begin{tabular}{@{}p{2.5cm}p{5cm}p{5cm}@{}}
\toprule
\textbf{Algorithm} & \textbf{Key Strengths} & \textbf{Main Limitations} \\
\midrule
\textbf{Q-Seg} & 
$\bullet$ True quantum execution available \newline
$\bullet$ Natural graph-cut formulation \newline
$\bullet$ Global optimization via tunneling &
$\bullet$ Limited to tiny images \newline
$\bullet$ Binary segmentation only \newline
$\bullet$ Embedding overhead significant \\
\midrule
\textbf{QuantumLOGISMOS} & 
$\bullet$ Finds multiple optimal solutions \newline
$\bullet$ Geometric constraints preserved \newline
$\bullet$ Medical imaging applications &
$\bullet$ Simulation-only currently \newline
$\bullet$ Qubit count limits scale \newline
$\bullet$ QAOA parameter optimization challenging \\
\midrule
\textbf{FFQOAK} & 
$\bullet$ No quantum hardware needed \newline
$\bullet$ Scales to large images \newline
$\bullet$ Significant quality improvement &
$\bullet$ Not true quantum algorithm \newline
$\bullet$ Limited theoretical guarantees \newline
$\bullet$ Hyperparameter tuning required \\
\midrule
\textbf{QCFFCM} & 
$\bullet$ Near-real-time performance \newline
$\bullet$ High-resolution image support \newline
$\bullet$ Practical deployment ready &
$\bullet$ Quantum simulation only \newline
$\bullet$ Manual threshold selection \newline
$\bullet$ Application-specific tuning \\
\bottomrule
\end{tabular}
\caption{Strengths and limitations of quantum segmentation algorithms}
\label{tab:strengths-limitations}
\end{table}

\subsubsection{Application Domain Suitability}

\begin{keyconceptbox}[Algorithm Selection Guidelines]
\textbf{For Medical Image Segmentation:}
\begin{itemize}
    \item Small ROIs with geometric constraints $\rightarrow$ QuantumLOGISMOS
    \item Large CT/MRI scans requiring speed $\rightarrow$ QCFFCM or FFQOAK
\end{itemize}

\textbf{For General Image Segmentation:}
\begin{itemize}
    \item Proof-of-concept on quantum hardware $\rightarrow$ Q-Seg
    \item Production deployment today $\rightarrow$ FFQOAK (quantum-inspired)
\end{itemize}

\textbf{For Scientific Imaging:}
\begin{itemize}
    \item High-resolution time-critical applications $\rightarrow$ QCFFCM
    \item Texture-rich SAR imagery $\rightarrow$ Q-Seg with MRF
\end{itemize}
\end{keyconceptbox}

\subsubsection{Future Outlook}

\paragraph{Near-Term (1-3 Years)}

\begin{itemize}
    \item Improved NISQ algorithms with better noise resilience
    \item Hybrid approaches dominating practical applications
    \item Quantum-inspired methods continuing to advance
\end{itemize}

\paragraph{Medium-Term (3-7 Years)}

\begin{itemize}
    \item Fault-tolerant quantum computers enabling larger problems
    \item Direct encoding of moderate-resolution images becoming feasible
    \item Quantum machine learning integration with segmentation
\end{itemize}

\paragraph{Long-Term (7+ Years)}

\begin{itemize}
    \item Full-scale quantum image processing possible
    \item Exponential speedups realized for suitable problem instances
    \item Quantum advantage demonstrated across broader image analysis tasks
\end{itemize}

\begin{tipbox}[Research Opportunities]
Key open problems for quantum image segmentation:
\begin{enumerate}
    \item Efficient quantum data loading for classical images
    \item Error mitigation strategies for segmentation accuracy
    \item Multi-class segmentation with limited qubit resources
    \item Benchmarking frameworks for fair algorithm comparison
\end{enumerate}
\end{tipbox}

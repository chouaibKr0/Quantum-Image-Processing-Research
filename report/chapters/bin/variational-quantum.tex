% chapters/algorithms/variational-quantum.tex
% Variational Quantum Clustering (QAOA, VQE)

\subsection{Variational Quantum Clustering}

Variational approaches use parameterized quantum circuits (PQCs) optimized through hybrid classical-quantum loops.

\subsubsection{Quantum Approximate Optimization Algorithm (QAOA)}

QAOA can be applied to clustering formulated as combinatorial optimization:

\begin{equation}
    \ket{\psi(\boldsymbol{\gamma}, \boldsymbol{\beta})} = \prod_{p=1}^{P} e^{-i\beta_p H_M} e^{-i\gamma_p H_C} \ket{s}
\end{equation}

where $H_C$ encodes the clustering objective and $H_M$ is a mixing Hamiltonian.

\subsubsection{Variational Quantum Eigensolver (VQE) for Clustering}

VQE can find the ground state of a Hamiltonian encoding the clustering problem:

\begin{enumerate}
    \item Define Hamiltonian $H$ such that its ground state encodes optimal clustering
    \item Prepare parameterized ansatz $\ket{\psi(\boldsymbol{\theta})}$
    \item Measure $\braket{H} = \braket{\psi(\boldsymbol{\theta})|H|\psi(\boldsymbol{\theta})}$
    \item Classically optimize $\boldsymbol{\theta}$ to minimize $\braket{H}$
\end{enumerate}

\begin{tipbox}
Variational methods are the most promising for near-term quantum devices because they can tolerate noise and work with limited qubit counts. They have been demonstrated on actual quantum hardware for small-scale clustering problems.
\end{tipbox}

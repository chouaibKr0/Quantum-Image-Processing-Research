% chapters/algorithms/quantum-kmeans.tex
% Quantum K-Means and Variants

\subsection{Quantum K-Means and Variants}

Quantum versions of K-means leverage quantum speedups in distance calculations and centroid updates.

\subsubsection{q-Means Algorithm}

The q-means algorithm, proposed by Kerenidis et al. (2019), achieves exponential speedup over classical K-means under certain conditions:

\begin{algorithm}
\caption{q-Means Algorithm (Simplified)}
\begin{algorithmic}[1]
\STATE \textbf{Input:} Quantum access to data matrix $V \in \mathbb{R}^{n \times d}$, number of clusters $k$
\STATE Initialize centroids using quantum sampling
\REPEAT
    \STATE Use quantum distance estimation to find nearest centroid for each point
    \STATE Update centroids using quantum linear algebra
\UNTIL{convergence}
\STATE \textbf{Output:} Cluster assignments
\end{algorithmic}
\end{algorithm}

\textbf{Complexity:} $O\left(k^2 d \frac{\eta^{2.5}}{\delta^2} \text{polylog}(nd)\right)$ per iteration, where $\eta$ is a condition number and $\delta$ is the desired precision.

\begin{warningbox}
Current quantum hardware (NISQ devices) has limited qubits and high error rates, which constrains practical implementations. The q-means algorithm requires fault-tolerant quantum computers with quantum RAM (qRAM), which are not yet available.
\end{warningbox}

\subsubsection{Variational Quantum K-Means}

For NISQ devices, variational approaches are more practical:

\begin{enumerate}
    \item Encode data points using parameterized quantum circuits
    \item Use a variational classifier to assign cluster labels
    \item Optimize circuit parameters using classical optimization
\end{enumerate}

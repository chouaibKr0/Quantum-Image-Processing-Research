% chapters/algorithms/comparative-analysis.tex
% Comparative Analysis of Quantum Clustering Algorithms

\subsection{Comparative Analysis}

Table \ref{tab:quantum-comparison} summarizes the key characteristics of quantum clustering algorithms for image segmentation.

\begin{table}[h]
\centering
\caption{Comparison of Quantum Clustering Algorithms}
\label{tab:quantum-comparison}
\begin{tabular}{@{}lccc@{}}
\toprule
\textbf{Algorithm} & \textbf{Speedup} & \textbf{Hardware} & \textbf{Maturity} \\
\midrule
q-Means & Exponential & Fault-tolerant + qRAM & Theoretical \\
Quantum Spectral & Exponential & Fault-tolerant & Theoretical \\
QAOA Clustering & Potential & NISQ & Experimental \\
VQE Clustering & Heuristic & NISQ & Experimental \\
Quantum Fuzzy & Quadratic & Fault-tolerant & Theoretical \\
%% Add new algorithms here following the same format:
%% Algorithm Name & Speedup Type & Hardware Req & Maturity Level \\
\bottomrule
\end{tabular}
\end{table}

\begin{keyconceptbox}[Current State of the Art]
While theoretical quantum algorithms promise significant speedups, practical quantum image segmentation remains limited by:
\begin{itemize}
    \item Data loading bottleneck: Encoding classical images into quantum states
    \item Hardware limitations: Qubit counts, error rates, and coherence times
    \item Scalability: Current demonstrations limited to small images
\end{itemize}
Variational methods on NISQ devices represent the most promising near-term approach.
\end{keyconceptbox}

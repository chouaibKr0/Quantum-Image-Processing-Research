% chapters/introduction.tex
% Introduction chapter

\chapter*{Introduction}
\addcontentsline{toc}{chapter}{Introduction}

Image segmentation stands as one of the most fundamental and challenging problems in computer vision and image processing. At its core, segmentation involves partitioning a digital image into multiple segments or regions, where each segment corresponds to meaningful objects or parts of objects. This process serves as a critical preprocessing step for numerous applications, including medical image analysis, autonomous driving, satellite imagery interpretation, and object recognition systems.

Traditionally, clustering algorithms have been the workhorses of image segmentation. Methods such as K-means, fuzzy C-means, and spectral clustering have demonstrated remarkable success in grouping pixels based on color, texture, or spatial proximity. However, as image resolutions increase and the demand for real-time processing grows, these classical approaches encounter significant computational bottlenecks. The curse of dimensionality, coupled with the need to process millions of pixels simultaneously, pushes classical algorithms to their limits.

\begin{keyconceptbox}
Quantum computing offers the potential to process information in fundamentally different ways than classical computers, potentially providing speedups for certain computational tasks including clustering. By leveraging quantum phenomena such as superposition and entanglement, quantum algorithms can explore exponentially large solution spaces more efficiently than their classical counterparts.
\end{keyconceptbox}

This report provides a comprehensive review of quantum clustering techniques and their application to image segmentation. We explore how quantum mechanical principles can be harnessed to overcome the computational limitations of classical methods, examining both theoretical foundations and practical implementations.

\section*{Objectives of This Review}

This report aims to:

\begin{itemize}
    \item Establish a solid foundation in classical clustering methods and their application to image segmentation
    \item Identify the computational bottlenecks that motivate the exploration of quantum approaches
    \item Review the state-of-the-art in quantum image representation and processing
    \item Provide a comparative analysis of quantum clustering algorithms for image segmentation
    \item Discuss current limitations and future research directions
\end{itemize}

\section*{Report Structure}

The remainder of this report is organized as follows:

\begin{description}
    \item[Chapter 1] presents the classical paradigm of image segmentation via clustering, covering fundamental definitions, classical algorithms, and computational challenges.
    \item[Chapter 2] explores quantum clustering as a new frontier, discussing quantum image representations and reviewing quantum segmentation algorithms.
    \item[Conclusion] summarizes key findings and outlines future research directions.
    \item[Appendix A] provides a primer on quantum information processing for readers less familiar with quantum computing concepts.
    \item[Appendix B] covers the fundamentals of fuzzy logic as applied to segmentation.
\end{description}

\documentclass[12pt]{article}
\usepackage{amsmath,amssymb,amsfonts}
\usepackage{graphicx}
\usepackage{hyperref}
\usepackage{booktabs}
\usepackage[margin=1in]{geometry}
\usepackage{algorithm}
\usepackage{algpseudocode}

\title{Technical Report: Quantum Computing-based Fast Fuzzy C-Means (QCFFCM) for Solar Coronal Hole Detection}
\author{}
\date{}

\begin{document}

\maketitle

\section{Introduction}
The detection of solar coronal holes (CHs) is critical for space weather prediction. Traditional image segmentation methods struggle with the high resolution (4096×4096) and volume of SDO/AIA solar imagery. This report presents a \textbf{quantum computing-based Fast Fuzzy C-Means (QCFFCM)} model that integrates quantum optimization into the FFCM algorithm. The method uses quantum approximate optimization to accelerate clustering while maintaining segmentation accuracy, achieving near-real-time performance (~12 seconds per image).

\section{Classical Fast Fuzzy C-Means (FFCM)}

\subsection{Traditional Fuzzy C-Means (FCM)}
The classical FCM minimizes:

\[
J_E = \sum_{j=1}^{N} \sum_{i=1}^{c} \mu_{ij}^m \| x_j - v_i \|^2
\tag{1}
\]

where:
\begin{itemize}
    \item $N$: number of pixels ($\approx16.8$ million for 4096×4096)
    \item $c$: number of clusters (2 for CH detection)
    \item $\mu_{ij}$: membership of pixel $j$ in cluster $i$
    \item $m$: fuzziness constant ($m = 2$ in implementation)
    \item $x_j$: intensity of pixel $j$
    \item $v_i$: centroid of cluster $i$
\end{itemize}

Centroid and membership updates:

\[
v_i = \frac{\sum_{j=1}^{N} \mu_{ij}^m x_j}{\sum_{j=1}^{N} \mu_{ij}^m}
\tag{2}
\]

\[
\mu_{ij} = \left[ \sum_{k=1}^{c} \left( \frac{\| x_j - v_i \|}{\| x_j - v_k \|} \right) \right]^{-\frac{1}{m-1}}
\tag{3}
\]

\subsection{Histogram-Based Acceleration (FFCM)}
To reduce computational complexity, FFCM operates on intensity histograms rather than individual pixels:

Define intensity occurrence frequency:

\[
n_L = \sum_{j=0}^{N} \delta[x_j - L], \quad L \in \{0,1,\dots,255\}
\tag{4}
\]

\[
\delta[x_j - L] = 
\begin{cases} 
1 & x_j = L \\
0 & x_j \neq L 
\end{cases}
\]

The FFCM energy function becomes:

\[
J_E = \sum_{L=0}^{255} \sum_{i=1}^{c} \mu_{iL}^m L \| n_L - v_i \|^2
\tag{5}
\]

Modified centroid and membership equations:

\[
v_i = \frac{\sum_{L=0}^{255} \mu_{iL}^m n_i L}{\sum_{L=0}^{255} \mu_{iL}^m}
\tag{6}
\]

\[
\mu_{iL} = \left[ \sum_{k=1}^{c} \left( \frac{\| L - v_i \|}{\| L - v_k \|} \right) \right]^{-\frac{1}{m-1}}
\tag{7}
\]

Convergence criterion:

\[
\| \hat{\mu}^{k+1} - \hat{\mu}^k \| < \epsilon
\tag{8}
\]

This reduces search space from $N$ to 256 intensity levels.

\section{Quantum Optimization of FFCM using 3-ADMM-H}

\subsection{Constrained Optimization Formulation}
The FFCM energy minimization is reformulated as:

\[
\begin{aligned}
\min_{\mu} & \quad \sum_{j=1}^{N} \sum_{i=1}^{c} \mu_{ij}^m \| x_j - v_i \|^2 \\
\text{subject to:} & \quad \sum_{j=1}^{N} \sum_{i=1}^{c} \mu_{ij}^m \| x_j - v_i \|^2 \geq 1 \\
& \quad v_i \geq 0, \quad i = 1, \dots, c \\
& \quad \sum_{i=1}^{c} v_i \geq 1
\end{aligned}
\tag{9}
\]

Optimal value:

\[
p^* = \inf \left\{ \sum_{j=1}^{N} \sum_{i=1}^{c} \mu_{ij}^m \| x_j - v_i \|^2 \;\middle|\; J_E \geq 1, v \in \Upsilon \right\}
\tag{10}
\]

\subsection{Three-Block ADMM Heuristic (3-ADMM-H)}
To handle inequality constraints, the problem is reformulated with auxiliary variable $z$:

\[
p^* = \inf \left\{ \sum_{j=1}^{N} \sum_{i=1}^{c} \mu_{ij}^m \| x_j - v_i \|^2 \;\middle|\; J_E - z = 1, v \in \Upsilon, z \in \mathfrak{R}_+^n \right\}
\tag{11}
\]

The \textbf{scaled augmented Lagrangian}:

\[
\begin{aligned}
L_\rho(\mu, z, y, v, \lambda, \eta) &= \sum_{j=1}^{N} \sum_{i=1}^{c} \mu_{ij}^m \| x_j - v_i \|^2 \\
&+ \rho \sum_{i,j} \lambda_{i,j} \left( \sum_{j=1}^{N} \sum_{i=1}^{c} \mu_{ij}^m \| x_j - v_i \|^2 - z - 1 \right) \\
&+ \frac{\rho}{2} \sum_{i,j} \left( \sum_{j=1}^{N} \sum_{i=1}^{c} \mu_{ij}^m \| x_j - v_i \|^2 - z - 1 \right)^2 \\
&+ \tau \rho \eta \left( \sum_{i=1}^{c} v_i - y - 1 \right) \\
&+ \frac{\tau \rho}{2} \left( \sum_{i=1}^{c} v_i - y - 1 \right)^2
\end{aligned}
\tag{12}
\]

where:
\begin{itemize}
    \item $\rho$: penalty parameter
    \item $\tau$: additional penalty parameter
    \item $\lambda, \eta$: Lagrange multipliers
    \item $y$: auxiliary variable for centroid constraints
\end{itemize}

\subsection{Quantum Subproblem via QAOA}
The 3-ADMM-H splits problem (12) into:
\begin{enumerate}
    \item \textbf{QUBO subproblem} (solved via QAOA on quantum device)
    \item \textbf{Convex optimization subproblem} (solved classically)
    \item \textbf{Convex-quadratic combination subproblem} (solved classically)
\end{enumerate}

The \textbf{QUBO subproblem} is solved using \textbf{Quantum Approximate Optimization Algorithm (QAOA)}:

Quantum state preparation:

\[
|\gamma, \beta\rangle = U(B, \beta_P) U(C, \gamma_P) \dots U(B, \beta_1) U(C, \gamma_1) | s \rangle
\tag{13}
\]

Problem Hamiltonian:

\[
U(C, \gamma) = e^{-i\gamma C} = \prod_{\alpha=1}^{M} e^{-i\gamma C_\alpha}
\tag{14}
\]

Mixing Hamiltonian:

\[
U(B, \beta) = e^{-i\beta B} = \prod_{t=1}^{k} e^{-i\beta \sigma_t^X}
\tag{15}
\]

where:

\[
B = \sum_{t=1}^{k} \sigma_t^X
\tag{16}
\]

Initial state (uniform superposition):

\[
| s \rangle = \frac{1}{\sqrt{2^k}} \sum_{\mathfrak{z}} | \mathfrak{z} \rangle
\tag{17}
\]

Parameter ranges:
\begin{itemize}
    \item $\gamma \in [0, 2\pi]$
    \item $\beta \in [0, \pi]$
\end{itemize}

\textbf{Multi-start strategy} is used to escape local optima.

\subsection{Classical Subproblems}
\begin{itemize}
    \item \textbf{Block 2}: Updates membership values $\mu_{ij}$ using convex optimization.
    \item \textbf{Block 3}: Updates centroids $v_i$ via quadratic programming.
\end{itemize}

\subsection{Iterative Hybrid Optimization}
Process:
\begin{enumerate}
    \item Initialize randomly
    \item Repeat until convergence:
    \begin{itemize}
        \item Solve QUBO via QAOA (quantum)
        \item Update $\mu_{ij}$ (classical)
        \item Update $v_i$ (classical)
        \item Update Lagrange multipliers
        \item Check convergence: $\| \hat{\mu}^{k+1} - \hat{\mu}^k \| < \epsilon$
    \end{itemize}
\end{enumerate}

\section{Post-Processing for Coronal Hole Extraction}
After QCFFCM segmentation:
\begin{itemize}
    \item \textbf{Foreground}: Potential CH regions (darker areas)
    \item \textbf{Background}: Non-CH regions
\end{itemize}

\textbf{Circular Hough Transform (CHT)} initializes a static contour to:
\begin{enumerate}
    \item Remove solar limbs
    \item Isolate disk interior
\end{enumerate}

\textbf{Classification rules}:
\begin{itemize}
    \item Foreground outside contour $\rightarrow$ background
    \item Background inside contour $\rightarrow$ background  
    \item Foreground inside contour $\rightarrow$ CH candidate
\end{itemize}

\textbf{Area-based morphological operation}:
\begin{itemize}
    \item Remove regions with area $< T_A$
    \item Threshold selection is critical and manual
    \item Future work will focus on optimal threshold selection
\end{itemize}

\section{Experimental Results}

\subsection{Dataset}
\begin{itemize}
    \item \textbf{365 SDO/AIA 193Å images} from 2017
    \item \textbf{Resolution}: 4096×4096
    \item \textbf{Ground truth}: NOAA/SWPC synoptic maps (SM-GT) and region-growing maps (RG-GT)
\end{itemize}

\subsection{Performance Metrics}
\textbf{F1 Score}:

\[
\text{F1} = \frac{2|A \cap B|}{|A| + |B|}
\tag{18}
\]

\textbf{Accuracy Rate}:

\[
\text{Accuracy} = \frac{|A|}{|B|}
\tag{19}
\]

where $A$ = detected region, $B$ = ground truth.

\subsection{Comparative Results}
\begin{itemize}
    \item \textbf{Execution time}: QCFFCM $\approx$ 12 seconds vs. 40--240 seconds for classical methods
    \item \textbf{F1 scores} (vs. RG-GT):
    \begin{itemize}
        \item February 2017: \textbf{0.9199} (highest)
        \item Annual average: \textbf{0.75--0.85}
        \item Comparable to CHIMERA and SPoCA
    \end{itemize}
    \item \textbf{Visual analysis}: QCFFCM captures CH boundary curvatures better than CNN, SPoCA, CHIMERA.
\end{itemize}

\subsection{System Specifications}
\begin{itemize}
    \item \textbf{CPU}: Intel Core i5-8250U @ 1.8GHz
    \item \textbf{RAM}: 4GB
    \item \textbf{OS}: Windows 10 64-bit
    \item \textbf{Quantum simulation}: Classical simulation of QAOA
\end{itemize}

\section{Conclusion}
The \textbf{Quantum Computing-based Fast Fuzzy C-Means (QCFFCM)} model successfully demonstrates the practical advantages of quantum-classical hybrid computing for solar image analysis. By integrating \textbf{Quantum Approximate Optimization Algorithm (QAOA)} with classical FFCM through a \textbf{3-ADMM-H} framework, the approach achieves \textbf{~12 seconds per image} processing time—3-20× faster than classical methods—while maintaining competitive accuracy (0.9199 F1 score). 

This quantum-enhanced acceleration enables \textbf{near-real-time coronal hole detection}, addressing a critical bottleneck in space weather monitoring. The methodology establishes a viable template for applying quantum optimization to large-scale scientific image segmentation, with the quantum component providing tangible speed advantages even in classical simulation, promising greater gains on future quantum hardware.

\end{document}
\documentclass[12pt]{article}
\usepackage{amsmath,amssymb,graphicx,hyperref,booktabs,geometry}
\geometry{a4paper,margin=1in}
\usepackage{enumitem}
\setlist[itemize]{leftmargin=*}

\title{Hybrid Quantum--Classical Models for Image Clustering: \\
A Technical Report on FFQOAK for COVID-19 CT Segmentation}

\begin{document}

\maketitle

\section{Introduction}
Image clustering is a fundamental task in medical image analysis, particularly for the segmentation of pathological regions in diagnostic imaging. The \textbf{K-means clustering (KMC)} algorithm is widely used due to its simplicity and efficiency. However, KMC suffers from two critical limitations:

\begin{enumerate}
    \item \textbf{Sensitivity to initial cluster centers} -- Different initializations can lead to significantly different clustering results.
    \item \textbf{Convergence to local optima} -- The algorithm often terminates at suboptimal solutions rather than finding the global optimum.
\end{enumerate}

These limitations become particularly problematic in medical image segmentation, where accurate delineation of pathological regions (such as COVID-19 lung infections) is crucial for diagnosis and treatment planning.

\textbf{Hybrid quantum--classical approaches} offer a promising solution to these challenges by combining the efficiency of classical algorithms with the global optimization capabilities of quantum-inspired methods. This report examines the \textbf{Fast Forward Quantum Optimization Algorithm combined with K-means (FFQOAK)} -- a novel hybrid method proposed for COVID-19 CT image segmentation. We focus on the mathematical foundations, implementation details, and advantages of this hybrid approach over traditional methods.

\section{Background: K-means Clustering and Its Limitations}
\subsection{Standard K-means Algorithm}
For an input image \(I_p\) with gray level values \(P_i \in [0, 255]\), the K-means algorithm partitions these values into \(\theta\) clusters by minimizing the within-cluster variance:

\[
J = \sum_{i=1}^{n} \sum_{z=1}^{\theta} \|P_i - C_z\|^2
\]

where:
\begin{itemize}
    \item \(P_i\) represents the gray level value of pixel \(i\)
    \item \(C_z\) represents the center of cluster \(z\)
    \item \(n\) is the total number of pixels
\end{itemize}

The algorithm proceeds iteratively through two steps:

\begin{enumerate}
    \item \textbf{Assignment Step:} Assign each pixel to the cluster with the nearest center:
    \[
    Z_z = \{P_i : \|P_i - C_z\|^2 \leq \|P_i - C_j\|^2 \ \forall j, 1 \leq j \leq \theta\}
    \]
    
    \item \textbf{Update Step:} Recalculate cluster centers as the mean of all points in the cluster:
    \[
    C_z = \frac{1}{|Z_z|} \sum_{P_i \in Z_z} P_i
    \]
\end{enumerate}

\subsection{Limitations of Standard K-means}
The algorithm's performance critically depends on:
\begin{itemize}
    \item \textbf{Initial cluster centers} -- Poor initialization leads to suboptimal clustering
    \item \textbf{Convergence to local minima} -- The algorithm terminates when no reassignments occur, which may represent only a local optimum
    \item \textbf{Fixed number of clusters} -- Requires prior knowledge of \(\theta\)
\end{itemize}

Previous hybrid approaches (GAK, PSOK, DPSOK, ACOK) attempted to address these issues by combining K-means with classical optimization algorithms, but these methods often either:
\begin{enumerate}
    \item Were computationally expensive
    \item Only partially overcame the limitations
    \item Introduced new hyperparameters requiring careful tuning
\end{enumerate}

\section{Quantum-Inspired Optimization: The FFQOA Framework}
\subsection{Quantum Mechanical Inspiration}
The Fast Forward Quantum Optimization Algorithm (FFQOA) draws inspiration from quantum mechanical principles, particularly:
\begin{itemize}
    \item \textbf{Quantum superposition} -- A quantum can exist in multiple states simultaneously
    \item \textbf{Quantum tunneling} -- The ability to escape local minima by passing through energy barriers
    \item \textbf{Wave function collapse} -- The probabilistic nature of quantum measurement
\end{itemize}

In FFQOA, optimization is modeled as a \textbf{quantum system} where:
\begin{itemize}
    \item Each potential solution is represented as a \textbf{``quantum''} (\(Q_k\))
    \item The collective system of quanta forms a \textbf{quantum system} (\(\mathbb{Q}\))
    \item Each quantum has properties analogous to physical quanta: \textbf{location}, \textbf{movement}, and \textbf{displacement}
\end{itemize}

\subsection{Mathematical Formulation of FFQOA}
\subsubsection{Quantum System Initialization}
The quantum system is initialized using a Schrödinger equation-inspired formulation:

\[
Q_k(e) = \phi \cdot Q1_k(e) + (1-\phi) \cdot Q2_k(e)
\]

where:
\begin{itemize}
    \item \(Q_k(e)\) represents the \(k\)-th quantum at epoch \(e\)
    \item \(\phi = a + ib\) is a complex number with \(a, b \in [0,1]\)
    \item \(Q1_k(e) = \{G^{UB} + r_1 \cdot (G^{UB} - G^{LB})\}\)
    \item \(Q2_k(e) = \{G^{LB} + r_2 \cdot (G^{UB} - G^{LB})\}\)
    \item \(G^{LB}\) and \(G^{UB}\) are the lower and upper bounds of the search space
    \item \(r_1, r_2\) are random numbers in \([0,1]\)
\end{itemize}

The use of complex numbers \(\phi\) introduces quantum superposition principles, allowing each quantum to explore multiple regions of the search space simultaneously in early iterations.

\subsubsection{Quantum Properties}
\textbf{Location} (\(L_k(e)\)): Represents the quantum's position in the search space:
\[
L_k(e) = \frac{1}{Q_k(e)} e^{-2/Q_k(e)}
\]

\textbf{Movement} (\(M_k(e)\)): Represents the quantum's tendency to change position:
\[
M_k(e) = |Q_k(e) - \frac{L_k(e)}{2} \ln(1/m_f)|
\]
where \(m_f \in [0,1]\) is the \textbf{quantum movement factor}.

\textbf{Displacement} (\(D_k(e)\)): The actual change in position resulting from movement:
\[
D_k(e) = 2 \cdot |L_k(e) - M_k(e)|
\]

\subsubsection{Search Scope Enhancement}
The key innovation of FFQOA is its \textbf{three-component search enhancement mechanism}:

\[
M_k(e+1) = M_1 + M_2 + M_3
\]

where:

\begin{enumerate}
    \item \textbf{Preceding Movement Component} (\(M_1\)):
    \[
    M_1 = \alpha \cdot M_k(e)
    \]
    where \(\alpha = \alpha_{max} - e \times \frac{|\alpha_{max} - \alpha_{min}|}{E}\) is the \textbf{quantum acceleration factor}.
    
    \item \textbf{Personal Network Component} (\(M_2\)):
    \[
    M_2 = \ln(1/m_f) \cdot r_3 \cdot [pBD_k(e) - D_k(e)]
    \]
    where \(pBD_k(e)\) is the \textbf{personal best displacement} -- the best position found by quantum \(k\) so far.
    
    \item \textbf{Global Network Component} (\(M_3\)):
    \[
    M_3 = \ln(1/m_f) \cdot r_4 \cdot [gBD(e) - D_k(e)]
    \]
    where \(gBD(e)\) is the \textbf{global best displacement} -- the best position found by any quantum in the system.
\end{enumerate}

The displacement is then updated:
\[
D_k(e+1) = D_k(e) + M_k(e+1)
\]

This formulation creates a balanced \textbf{exploration-exploitation tradeoff}:
\begin{itemize}
    \item \(M_1\) maintains momentum (exploitation)
    \item \(M_2\) leverages individual quantum's history (local exploitation)
    \item \(M_3\) incorporates global information (exploration)
\end{itemize}

\section{The Hybrid FFQOAK Method}
\subsection{Integration Strategy}
The hybrid FFQOAK method integrates FFQOA with K-means through a \textbf{two-phase optimization framework}:

\textbf{Phase 1 (Quantum Optimization):} FFQOA searches for optimal cluster centers in the continuous space of possible centers.

\textbf{Phase 2 (Classical Clustering):} K-means performs pixel assignment based on the centers provided by FFQOA.

\subsection{Mathematical Integration}
The integration is achieved by:
\begin{enumerate}
    \item Representing cluster centers as quantum displacements
    \item Using the K-means objective function as the fitness function for FFQOA
    \item Iteratively refining centers through quantum optimization
\end{enumerate}

Specifically:

Let \(C = [C_1, C_2, ..., C_\theta]\) be the cluster centers. In FFQOAK:
\begin{itemize}
    \item Each set of potential centers is represented as a quantum displacement \(D_j(e)\)
    \item The fitness of displacement \(D_j(e)\) is evaluated using the K-means objective function:
    \[
    \hat{J}_j(e) = \sum_{i=1}^n \min_{z} \|P_i - X_z\|^2, \quad \forall X_z \in D_j(e)
    \]
    \item FFQOA optimizes \(D_j(e)\) to minimize \(\hat{J}_j(e)\)
\end{itemize}

\subsection{Algorithm Steps}
The complete FFQOAK algorithm proceeds as follows:

\begin{enumerate}
    \item \textbf{Input:} Image \(I_p\) with gray level values \(G_{ld} = \{P_1, P_2, ..., P_n\}\)
    
    \item \textbf{Initialize:} \(\theta\) clusters, \(q\) quanta with random displacements
    
    \item \textbf{For each epoch \(e = 1\) to \(E\):}
    \begin{itemize}
        \item For each quantum \(Q_k(e)\):
        \begin{itemize}
            \item \textbf{Assignment:} Assign each pixel \(P_i\) to nearest center in \(D_j(e)\) using Euclidean distance
            \item \textbf{Fitness Evaluation:} Compute \(\hat{J}_j(e)\) using Equation 24
            \item \textbf{Update Personal Best:} \(pBD_k(e)\) if \(\hat{J}_j(e)\) improves
            \item \textbf{Update Global Best:} \(gBD(e)\) if \(\hat{J}_j(e)\) is best among all quanta
        \end{itemize}
        \item \textbf{Quantum Movement Update:} Compute \(M_k(e+1)\) using Equation 15
        \item \textbf{Displacement Update:} Compute \(D_k(e+1)\) using Equation 20
    \end{itemize}
    
    \item \textbf{Output:} Optimal cluster centers from \(gBD(E)\), segmented image \(I_s\)
\end{enumerate}

\subsection{Parameter Settings}
Based on the research paper, optimal parameters for COVID-19 CT segmentation are:
\begin{itemize}
    \item Number of clusters: \(\theta = 3\)
    \item Number of quanta: \(q = 20\)
    \item Quantum movement factor: \(m_f = 0.14\)
    \item Acceleration factors: \(\alpha_{min} = 0.2\), \(\alpha_{max} = 0.9\)
    \item Maximum epochs: \(E = 100\)
\end{itemize}

\section{Advantages of the Hybrid Approach}
\subsection{Overcoming K-means Limitations}
\begin{enumerate}
    \item \textbf{Escape from Local Optima:} The quantum tunneling effect (modeled through the complex number formulation and stochastic components) allows FFQOAK to escape poor local minima that trap standard K-means.
    
    \item \textbf{Reduced Sensitivity to Initialization:} By starting with multiple quanta (potential solutions) and evolving them through quantum operations, FFQOAK explores a broader solution space, reducing dependence on initial cluster centers.
    
    \item \textbf{Balanced Exploration-Exploitation:} The three-component movement update (\(M_1\), \(M_2\), \(M_3\)) ensures both:
    \begin{itemize}
        \item \textbf{Exploration:} Searching new regions of the solution space (via \(M_3\))
        \item \textbf{Exploitation:} Refining promising solutions (via \(M_1\) and \(M_2\))
    \end{itemize}
\end{enumerate}

\subsection{Computational Advantages}
\begin{enumerate}
    \item \textbf{Parallel Exploration:} The quantum system inherently explores multiple solutions simultaneously, akin to quantum superposition.
    
    \item \textbf{Adaptive Search:} The acceleration factor \(\alpha\) decreases linearly with epochs, transitioning from exploration (early epochs) to exploitation (later epochs).
    
    \item \textbf{Information Sharing:} The global best component (\(M_3\)) enables rapid propagation of good solutions across the entire quantum system.
\end{enumerate}

\subsection{Performance Improvements}
The research paper demonstrates significant improvements over traditional methods:

\begin{table}[h]
\centering
\begin{tabular}{@{}lcccc@{}}
\toprule
Metric & K-means & Best Hybrid (ACOK) & FFQOAK & Improvement \\
\midrule
MSE & 5293.23 & 1136.73 & \textbf{712.30} & 37.3\% reduction \\
PSNR & 11.02 dB & 17.60 dB & \textbf{19.61 dB} & 11.4\% improvement \\
Jaccard & 0.37 & 0.84 & \textbf{0.90} & 7.1\% improvement \\
Convergence & Local optima & Near global & \textbf{Global optima} & Better exploration \\
\bottomrule
\end{tabular}
\end{table}

\section{Experimental Validation}
\subsection{Dataset and Preprocessing}
The method was validated on \textbf{COVID-19 CT scan images} from a publicly available dataset (Jun et al., 2020). Images were preprocessed using:
\begin{enumerate}
    \item \textbf{Adaptive filtering} for noise reduction
    \item \textbf{Histogram equalization} for contrast enhancement
\end{enumerate}

\subsection{Comparative Analysis}
FFQOAK was compared against:
\begin{enumerate}
    \item \textbf{Standard K-means}
    \item \textbf{Hybrid methods:} GAK, PSOK, DPSOK, ACOK
    \item \textbf{Performance metrics:} MSE, PSNR, Jaccard Similarity, Correlation Coefficient
\end{enumerate}

\subsection{Key Findings}
\begin{enumerate}
    \item \textbf{Superior Segmentation Accuracy:} FFQOAK achieved the best scores across all metrics, particularly in accurately delineating infected lung regions.
    
    \item \textbf{Faster Convergence:} The convergence curves (Figures 12-15 in the paper) show FFQOAK reaches near-optimal solutions in fewer iterations compared to other hybrid methods.
    
    \item \textbf{Robustness:} The same parameter set worked effectively across four different experimental sets, demonstrating robustness to image variations.
\end{enumerate}

\section{Conclusion and Future Directions}
The \textbf{FFQOAK hybrid method} successfully addresses the fundamental limitations of K-means clustering by integrating it with a quantum-inspired optimization algorithm. Key contributions include:

\begin{enumerate}
    \item \textbf{Novel Quantum Formulation:} A mathematically rigorous model inspired by quantum mechanics that effectively balances exploration and exploitation.
    
    \item \textbf{Effective Integration:} A seamless integration strategy where quantum optimization guides cluster center selection while classical K-means handles pixel assignment.
    
    \item \textbf{Practical Validation:} Demonstrated effectiveness on the challenging task of COVID-19 CT image segmentation.
\end{enumerate}

\textbf{Future research directions:}
\begin{enumerate}
    \item \textbf{Extension to Other Modalities:} Application to X-ray, MRI, and other medical imaging modalities.
    \item \textbf{Adaptive Parameter Tuning:} Development of self-adapting parameter adjustment mechanisms.
    \item \textbf{Quantum Hardware Implementation:} Exploration of actual quantum computing hardware to potentially accelerate the optimization process.
    \item \textbf{Multi-objective Optimization:} Extension to handle multiple conflicting objectives in image segmentation.
\end{enumerate}

The success of FFQOAK demonstrates the significant potential of \textbf{hybrid quantum--classical approaches} for complex optimization problems in medical image analysis and beyond.

\begin{thebibliography}{9}
\bibitem{singh2021}
Singh, P., \& Bose, S. S. (2021). A quantum-clustering optimization method for COVID-19 CT scan image segmentation. \textit{Expert Systems with Applications, 185}, 115637.

\bibitem{jun2020}
Jun, M., et al. (2020). COVID-19 CT lung and infection segmentation dataset.

\bibitem{huang2021}
Huang, Y.-P., Singh, P., Kuo, W.-L., \& Chu, H.-C. (2021). A type-2 fuzzy clustering and quantum optimization approach for crops image segmentation. \textit{International Journal of Fuzzy Systems, 23}(3), 615-629.

\bibitem{macqueen1967}
MacQueen, J. (1967). Some methods for classification and analysis of multivariate observations. \textit{Proceedings of the Fifth Berkeley Symposium on Mathematical Statistics and Probability}, 1, 281-297.
\end{thebibliography}

\end{document}